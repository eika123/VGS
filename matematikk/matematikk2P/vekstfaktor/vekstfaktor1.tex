\documentclass[a4, 11pt, twoside]{article}

\usepackage{pstricks,pst-plot}
\usepackage[utf8]{inputenc}
\usepackage[english,norsk]{babel}

\usepackage{ulem}

\usepackage{listings}
\usepackage{color}
\usepackage{xcolor}

\usepackage{natbib}


\lstset{ %
  backgroundcolor=\color{white},   % choose the background color; you must add \usepackage{color} or \usepackage{xcolor}
  basicstyle=\footnotesize,        % the size of the fonts that are used for the code
  breakatwhitespace=false,         % sets if automatic breaks should only happen at whitespace
  breaklines=true,                 % sets automatic line breaking
  %captionpos=b,                   % sets the caption-position to bottom
  commentstyle=\color{blue},       % comment style
  deletekeywords={...},            % if you want to delete keywords from the given language
  escapeinside={\%*}{*)},          % if you want to add LaTeX within your code
  extendedchars=true,              % lets you use non-ASCII characters; for 8-bits encodings only, does not work with UTF-8
  keepspaces=true,                 % keeps spaces in text, useful for keeping indentation of code (possibly needs columns=flexible)
  keywordstyle=\color{orange},       % keyword style
  language=Python,                 % the language of the code
  morekeywords={*,...},            % if you want to add more keywords to the set
  rulecolor=\color{black},         % if not set, the frame-color may be changed on line-breaks within not-black text (e.g. comments (green here))
  showspaces=false,                % show spaces everywhere adding particular underscores; it overrides 'showstringspaces'
  showstringspaces=false,          % underline spaces within strings only
  showtabs=false,                  % show tabs within strings adding particular underscores
  stepnumber=2,                    % the step between two line-numbers. If it's 1, each line will be numbered
  stringstyle=\color{violet},      % string literal style
  tabsize=2,                       % sets default tabsize to 2 spaces
}

\usepackage{caption}
\DeclareCaptionFont{white}{\color{white}}
\DeclareCaptionFormat{listing}{\colorbox{gray}{\parbox{\textwidth}{#1#2#3}}}
\captionsetup[lstlisting]{format=listing,labelfont=white,textfont=white}

% This concludes the preamble

\usepackage[colorlinks=true,citecolor=red]{hyperref}

% \usepackage[norsk]{babel}
\usepackage{amsmath, amssymb, amsthm}
\usepackage{makeidx}
\usepackage{graphicx}

\usepackage{fancyhdr}
\pagestyle{fancy}

\usepackage{accents}
\newcommand{\interior}[1]{\accentset{\smash{\raisebox{-0.12ex}{$\scriptstyle\circ$}}}{#1}\rule{0pt}{2.3ex}}
\fboxrule0.0001pt \fboxsep0pt

% used for diagrams, not needed here
% \usepackage{tikz}
% \usepackage{tikz-cd}
% \usetikzlibrary{matrix, arrows, decorations}



%%% New environments
\newtheorem{theorem}{Theorem}[section]
\newtheorem{lemma}[theorem]{Lemma}
\newtheorem{corollary}[theorem]{Corollary}
\newtheorem{prop}[theorem]{Proposition}

\theoremstyle{definition}
\newtheorem{defn}[theorem]{Definition}
\newtheorem{example}[theorem]{Example}
\newtheorem{eksempel}{Eksempel}
\setcounter{eksempel}{0}
\newtheorem{exercise}[theorem]{Exercise}
\newtheorem{remark}[theorem]{Remark}
\newtheorem{nremark}[theorem]{Bemerkning}
\newtheorem{advarsel}[theorem]{ADVARSEL}
\newtheorem{question}[theorem]{Question}
\newtheorem{conjecture}[theorem]{Conjecture}
\newtheorem{improvement}[theorem]{Improvement}
\newtheorem{discus}[theorem]{Discus}
\newtheorem{ptheorem}[theorem]{Possible Theorem}
\newtheorem{project}[theorem]{Project}
\newtheorem{solution}[theorem]{Solution}
\newcommand\hra{\hookrightarrow}

%%% Custom definitions and macros


\DeclareMathOperator{\vspan}{Span}
\renewcommand{\d}{\mathrm{\; d}} % for differensialet i integraler.

\newcommand{\Aut}{\mathop{{\rm Aut}}}
\newcommand{\End}{\mathop{{\rm End}}}
\newcommand{\Hom}{\mathop{{\rm Hom}}}
\newcommand{\rank}{\mathop{{\rm rank}}}
\newcommand{\st}{\text{ s.t }}
\renewcommand{\div}{\mathop{{\rm div}}}

\DeclareMathOperator{\Dx}{\frac{\d}{\d x}}
\DeclareMathOperator{\DDx}{\frac{\d^2}{\d x^2}}
\DeclareMathOperator{\erf}{erf}
\DeclareMathOperator{\erfc}{erfc}
\DeclareMathOperator{\sign}{sgn}


% Caligraphic letters
\newcommand{\cA}{\mathcal{A}}
\newcommand{\cB}{\mathcal{B}}
\newcommand{\cR}{\mathcal{R}}
\newcommand{\cC}{\mathcal{C}}
\newcommand{\cD}{\mathcal{D}}
\newcommand{\cE}{\mathcal{E}}
\newcommand{\cF}{\mathcal{F}}
\newcommand{\cG}{\mathcal{G}}
\newcommand{\cH}{\mathcal{H}}
\newcommand{\cI}{\mathcal{I}}
\newcommand{\cJ}{\mathcal{J}}
\newcommand{\cK}{\mathcal{K}}
\newcommand{\cL}{\mathcal{L}}
\newcommand{\cN}{\mathcal{N}}
\newcommand{\cO}{\mathcal{O}}
\newcommand{\cS}{\mathcal{S}}
\newcommand{\cZ}{\mathcal{Z}}
\newcommand{\cP}{\mathcal{P}}
\newcommand{\cT}{\mathcal{T}}
\newcommand{\cU}{\mathcal{U}}
\newcommand{\cV}{\mathcal{V}}
\newcommand{\cX}{\mathcal{X}}
\newcommand{\cW}{\mathcal{W}}

\newcommand{\A}{\mathbb{A}}
\newcommand{\B}{\mathbb{B}}
\newcommand{\C}{\mathbb{C}}
\newcommand{\D}{\mathbb{D}}
\renewcommand{\H}{\mathbb{H}}
\newcommand{\N}{\mathbb{N}}
\newcommand{\K}{\mathbb{K}}
\newcommand{\Q}{\mathbb{Q}}
\newcommand{\Z}{\mathbb{Z}}
\renewcommand{\P}{\mathbb{P}}
\newcommand{\R}{\mathbb{R}}
\newcommand{\U}{\mathbb{U}}
\newcommand{\bT}{\mathbb{T}}
\newcommand{\bD}{\mathbb{D}}

\newcommand{\vu}{\boldsymbol{u}}
\newcommand{\vv}{\boldsymbol{v}}
\newcommand{\vn}{\boldsymbol{n}}
\newcommand{\vf}{\boldsymbol{f}}

\def\di{\partial}
\def\bs{\backslash}
\def\e{\epsilon}
\def\la{\langle}
\def\ra{\rangle}



\renewcommand{\d}{\;\text{d}}
\newcommand{\ocM}{\overline{\mathcal{M}}^{+}}
\newcommand{\ocB}{\overline{\mathcal{B}}}
\newcommand{\ocA}{\overline{\mathcal{A}}}
\newcommand{\Lp}{\text{L}}
\newcommand{\aev}{\text{ a.e }}
\newcommand{\aeev}{\text{-a.e }}

\newcommand{\llra}{\xrightarrow{\hspace*{1cm}}}

\DeclareMathOperator*{\supp}{supp}
\DeclareMathOperator*{\essup}{essup}


\fancyhead[LE, RO]{Vekstfaktor}
\fancyhead[RE, LO]{Prosentregning}
\setlength{\headheight}{14pt}

\usepackage{exercise}
\def\ExerciseName{Oppgave}

\begin{document}



\section{Kort hjernetrim}
\begin{Exercise}
  Et selskap gjør forsøk på en rekke datamaskiner. $\frac{3}{4}$ av datamaskinene
  hadde feil og mangler. Av datamaskinene med feil og mangler var $\frac{1}{7}$ alvorlige.
\begin{itemize}
\item[\bf a)] Hvor mange prosent av datamaskinene hadde feil og mangler?
\item[\bf b)] Hvor stor (total) andel av datamaskinene hadde alvorlige feil og mangler? Oppgi svaret som
  brøk, prosent og desimaltall.
\item[\bf c)] Hva vil du si er sannsynligheten for at en datamaskin hadde alvorlige feil og mangler?
\end{itemize}
\end{Exercise}


\section{Vekstfaktor}
\subsection*{Kort teori}
En størrelse som øker med $p\%$ gir en vekstfaktor
\[V = 1 + \frac{p}{100}. \]
En størrelse som minker med $p\%$ gir en vekstfaktor
\[V = 1 - \frac{p}{100} \]
Tallet \[\frac{p}{100} \] kalles ofte \textbf{prosentfaktor}.
\begin{eksempel}
I 2008 var befolkningen i en kommune 22000. Fra 2008 til 2009 økte befolkningen
med 5\%. Dette tilsvarer en vekstfaktor på 
\[V = 1 + 0,05 = 1,05.\] I 2009 var da befolkningen gitt ved formelen
\[ N = GV = 22000\cdot1,05 = 23100 \]
\end{eksempel}

\begin{eksempel}
  En vare på 340kr går ned 20\%. Siden 20\% som desimaltall er 0,20, 
  får vi en vekstfaktor på
  1 - 0,20 = 0,80.
  Den nye prisen blir da \[N = GV = 340\text{kr}\cdot0,80 = 272\text{kr}. \]
  I formelen \[N = GV \] er $N$ ny verdi, $G$ er gammel verdi (altså grunnverdi),
  og $V$ er vekstfaktoren.
\end{eksempel}

\begin{Exercise}
Fyll inn det som mangler.
\newline
\begin{center}
\begin{tabular}{| l | l |}
\hline
\textbf{Endring i prosent} & \textbf{Vekstfaktor} \\ \hline  
+ 5\%                      & 1,05                 \\ \hline 
- 7\%                      & 0,93                 \\ \hline 
+ 7,6\%                    & 1,076                \\ \hline 
+ 7\%                      &                      \\ \hline 
+ 7,5\%                    &                      \\ \hline 
+ 17\%                     &                      \\ \hline 
- 37\%                     &                      \\ \hline 
                           & 0,80                 \\ \hline 
                           & 1,345                \\ \hline 
                           & 0,782                \\ \hline 
                           & 0,759                \\ \hline 
                           & 2,67                 \\ \hline 
                           & 3,345                 \\ \hline 
                           & 1,231                 \\ \hline 
0,5\%                      &                      \\ \hline
                           & 1,005                 \\ \hline 
\end{tabular}
\end{center}
\end{Exercise}

\begin{Exercise}\label{anbefalt3}
Lise kjøpte en jakke på 20\% avslag og betalte 1750kr.
Hva kostet jakken opprinnelig? Gi to løsningsmåter, en med
veien om 1, og en med vekstfaktor.
\end{Exercise}

\begin{Exercise}
\textit{V}\newline
  En vare kostet opprinnelig 1200kr, og ble satt ned 20\%. 
\begin{description}
  \item[a)] Hva er vekstfaktoren?
  \item[b)] Hva er gammel (ordinær) pris ?
  \item[c)] Forklar hva de ulike størrelsene betyr i formelen \[N = GV.\]
  \item[d)] Sett opplysningene du har inn i formelen over, og regn ut ny verdi.
\end{description}
\end{Exercise}

\begin{Exercise}
Vekstfaktorene for prisen på en vare er listet opp
i tabellen under. \newline
\begin{center}
\begin{tabular}{| l | l | l | l | l | l | l | l |}
\hline
2008 & 2009   & 2010   & 2011   & 2012   & 2013   & 2014   & 2015 \\ \hline
1,03 & 1,0255 & 1,024 & 1,02   & 1,035  & 1,027   & 0,978  & 1,01 \\ \hline
\end{tabular}
\end{center}
\begin{description}
\item[a)] Hva forteller tabellen deg om de årlige prisendringene i prosent?
\item[b)] Hvis varen kostet 50kr i begynnelsen av 2008, 
hvor mange prosent endret varen seg totalt? 
\item[c)] Hva er produktet av vekstfaktorene? Hvilken prosentivs vekst
svarer den til?
\item[d)] Sammenlign svarene i b) og c), og gi en tolkning eller forklaring
av fenomenet.
\end{description}
\end{Exercise}

\begin{Exercise}
  En vare gikk ned fra 3600kr til 2950kr.
  \begin{description}
    \item[a)] Sett opp formelen som beskriver sammenhengen mellom ny og gammel pris vha vekstfaktoren.
    \item[b)] Sett inn opplysningene du har inn i formelen over.
    \item[c)] Løs likningen du får.
    \item[d)] Hva er vekstfaktoren?
    \item[e)] Hvor mange prosent gikk prisen ned med ?
  \end{description}
\end{Exercise}

\begin{eksempel}[Finne gammel verdi:]
Jens kjøpte en bruktbil, og betalte 150000. Han regnet med at bilen har sunket
45\% i pris siden den var ny. Dette tilsvarer en vekstfaktor på $1 - 0,45 = 0,55$.
For å regne ut gammel pris setter vi dette inn i formelen
\begin{align*}
  N & = GV \\
  150000\text{kr} & = G\cdot0,55     \;\;\;\;\;\;\;\;\;\;\;\;\;\;\;\; \text{   setter inn $ N = 150000$kr og $V = 0,55$ } \\
  G & = \frac{150000\text{kr}}{0,55} \;\;\;\;\;\;\;\;\;\;\;\;\;\;\;\; \text{deler på $0,55$ på hver side } \\
  G & = 272727\text{kr}
\end{align*}
\end{eksempel}

\begin{Exercise}
  En vare med ordinær pris 5000kr gikk ned 30\%. Bruk vekstfaktor og regn ut ny pris.
\end{Exercise}

\begin{Exercise}
  En vare med ordinær pris 5000kr gikk ned 30\%. Bruk vekstfaktor og regn ut ny pris.
\end{Exercise}


\begin{Exercise}
  Petter betalte 1500 kr for en vare på 33\% tilbud. 
  \begin{description}
    \item[a)] Bruk veien om 1 til å regne ut ordinær pris
    \item[b)] Bruk vekstfaktor til å regne ut ordinær pris
  \end{description}
\end{Exercise}


\begin{Exercise}
  Petter betalte 1230 kr for en vare på 25\% tilbud. 
  \begin{description}
    \item[a)] Bruk veien om 1 til å regne ut ordinær pris
    \item[b)] Bruk vekstfaktor til å regne ut ordinær pris
    \item[c)] Hvilken måte er enklest (minst regning) ?
  \end{description}
\end{Exercise}



\begin{Exercise}
  Petter betalte 1230 kr for en vare på 25\% tilbud. 
  \begin{description}
    \item[a)] Bruk veien om 1 til å regne ut ordinær pris
    \item[b)] Bruk vekstfaktor til å regne ut ordinær pris
    \item[c)] Hvilken måte er enklest (minst regning) ?
  \end{description}
\end{Exercise}


\begin{Exercise}
  En vare gikk ned fra 2345kr til 1650kr.
  Hvor mange prosent gikk varen ned med? (Bruk vekstfaktor)
\end{Exercise}

\begin{Exercise}
  En vare på 530kr gikk først opp 10\% og deretter 25\%.
  \begin{description}
    \item[a)] Hva er vekstfaktoren for hver av prisendringene ?
    \item[b)] Hva er ny pris?
    \item[c)] Hvor mange prosent gikk varen opp med totalt ?
    \item[d)] Hva er produktet av vekstfaktorene, hvor mange prosent oppgang i prisen tilsvarer dette produktet? Sammenlign
      med svaret i c)
  \end{description}
\end{Exercise}


\begin{Exercise}
Prisen på en vare gikk først opp 10\% og deretter 25\%.
  \begin{description}
    \item[a)] Hva er vekstfaktoren for hver av prisendringene ?
    \item[b)] Hvor mange prosent gikk varen opp med totalt ?
  \end{description}
\end{Exercise}


\begin{Exercise}
  En gikk først opp 12,5\% så 20\% og deretter ned 5\%.
  \begin{description}
    \item[a)] Hva er vekstfaktoren for hver av prisendringene ?
    \item[b)] Hvor mange prosent gikk varen opp eller ned med totalt ?
  \end{description}
\end{Exercise}

\begin{Exercise}
\begin{description}
\item[a)] Jens har spart i aksjer i 10 år. Han har hatt en årlig avkastning
på 15\%. Hvor stor har avkastningen vært totalt ?
\item[b)] Anta at han nå har 500 000 kr. Hvor mange kroner hadde Jens når
han begynte å spare ?
\item[c)] Hva hvis han haddde 5 000 000 kr ?
\end{description}
\end{Exercise}




\end{document}


