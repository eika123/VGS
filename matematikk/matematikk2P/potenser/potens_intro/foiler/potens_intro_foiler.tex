\documentclass{beamer}

\usepackage[utf8]{inputenc}
\usepackage[norsk]{babel}

\usepackage{qtree}

\usepackage{amsmath, amssymb}
\usepackage[misc]{ifsym}
%%% New environments

%\newtheorem{theorem}{Theorem}[section]
%\newtheorem{lemma}[theorem]{Lemma}
%\newtheorem{corollary}[theorem]{Corollary}
%\newtheorem{prop}[theorem]{Proposition}

\newtheorem{teorem}{Teorem}[section]
\newtheorem{nlemma}[theorem]{Lemma}
\newtheorem{korollar}[theorem]{Korollar}
\newtheorem{setn}[theorem]{Setning}

\theoremstyle{definition}
\newtheorem{defn}[theorem]{Definition}
\newtheorem{ndefn}[theorem]{Definisjon}
%\newtheorem{example}[theorem]{Example} \newtheorem{eksempel}[theorem]{Eksempel}
\newtheorem{exercise}[theorem]{Exercise}
\newtheorem{remark}[theorem]{Remark}
\newtheorem{nremark}[theorem]{Bemerkning}
\newtheorem{question}[theorem]{Question}
\newtheorem{conjecture}[theorem]{Conjecture}
\newtheorem{improvement}[theorem]{Improvement}
\newtheorem{discus}[theorem]{Discus}
\newtheorem{ptheorem}[theorem]{Possible Theorem}
\newtheorem{project}[theorem]{Project}
%\newtheorem{solution}[theorem]{Solution}

\title{Potensregning}
\subtitle{Introduksjon - hva er en potens?}
\date{\today}
\subject{Mathematics}

%\usetheme{Warsaw}
%\usetheme{Antibes}
%\usetheme{Bergen}
%\usetheme{Berkeley}
%\usetheme{Berlin}
%\usetheme{Copenhagen}
%\usetheme{Darmstadt}
%\usetheme{Dresden}
%\usetheme{Frankfurt}
%\usetheme{Goettingen}
%\usetheme{Hannover}
%\usetheme{Ilmenau}
%\usetheme{JuanLesPins}
%\usetheme{Luebeck}
%\usetheme{Madrid}
%\usetheme{Malmoe}
%\usetheme{Marburg}
%\usetheme{Montpellier}
\usetheme{PaloAlto}
%\usetheme{Pittsburgh}
%\usetheme{Rochester}
%\usetheme{Singapore}
%\usetheme{Szeged}
%\usetheme{boxes}
%\usetheme{default}
%\usetheme{CambridgeUS}

\begin{document}

\frame{\titlepage}


\section{Potenser}

\begin{frame}
\frametitle{Eksempel: Grunntall og eksponent}
\huge\[2^3 = 2\cdot2\cdot2 \]
\normalsize
Viktige begreper
\begin{itemize}
\item<2-> Grunntall: 2
\item<3-> Eksponent: 3
\item<4-> Merk forskjell på $\displaystyle 2^3$ og $\displaystyle 2\cdot3$
\item<5->
Oppgave: Regn ut:
\begin{center}
\begin{tabular}{l l l l l l}
\textbf{a)} $2^4$ &
\textbf{b)} $3^2$ &
\textbf{c)} $3^3$ &
\textbf{d)} $6^2$
\end{tabular}
\end{center}
\end{itemize}
\end{frame}

\begin{frame}
\frametitle{Hva er grunntallet, og hva er eksponenten?}
Oppgave: Hva blir fortegnet ? 
\huge\[5^2 = ... \]
\huge\[(-5)^2 = ... \]
\huge\[-5^2 = ... \]
\end{frame}


\begin{frame}
\frametitle{Hva blir fortegnet ?}
Oppgave: 
\huge\[-(-5)^3 = ... \]
\end{frame}

\begin{frame}
\frametitle{Eksempel: Negative grunntall}
\huge\[(-2)^3 = (-2)\cdot(-2)\cdot(-2) = ... \]
\huge\[(-2)^4 = (-2)\cdot(-2)\cdot(-2)\cdot(-2) = ... \]
\normalsize
\begin{itemize}
\item<2-> Kan dere på forhånd si om en potens med negativt grunntall blir positiv eller negativ ?
\item<3-> Oppgave: Regn ut:
\begin{center}
\begin{tabular}{l l l l l}
\textbf{a)} $(-3)^2$ &
\textbf{b)} $(-3)^3$ &
\textbf{c)} $(-6)^2$ &
\textbf{d)} $(-6)^3$
\end{tabular}
\end{center}
\end{itemize}
\end{frame}

\begin{frame}
\frametitle{Eksempel: vanlige misforståelser}
Hva blir:
\huge
\begin{align*}
 (-2)^4 = &\\
 -2^4 = & \\
 (-2)^3 = &\\
 -(-2)^3 = &
\end{align*}
\normalsize
\end{frame}

\begin{frame}
\frametitle{Skrive tall som potens}
Skriv som potens med 2 som grunntall:
\begin{align*}
\textbf{a) } 4   & \\ \\
\textbf{b) } 8   & \\ \\
\textbf{c) } 64  &
\end{align*}
\end{frame}

\begin{frame}
\frametitle{Oppgave: Regn ut}
\huge
\begin{align*}
\textbf{a) } 3\cdot(5 - 9)^3 + 5^2 = &\\ \\
\textbf{b) } 50 - 3\cdot(5 - 7)^3 - 5^2 = & \\ \\
\textbf{c) } 50 - 4\cdot(5 - 7)^3 - (-5)^2 = &
\end{align*}
\normalsize
\end{frame}

\begin{frame}
Hvordan kan dere skrive dette som en potens med ett grunntall?
\huge\[3^4\cdot3^2\]
\huge\[3^{2012}\cdot3^4\]
\end{frame}

\begin{frame}
\frametitle{Kontrollspørsmål}
Skriv som en toerpotens:
\huge\[2^4\cdot2^2\]
\huge\[2^{2012}\cdot2^4\]
\end{frame}

\begin{frame}
Hvordan kan dere skrive dette som et produkt av to potenser ?
\huge\[(6\cdot5)^3\]
\huge\[(6\cdot5)^{2016}\]
\huge\[(a\cdot b)^{n}\]
\end{frame}

\end{document}
