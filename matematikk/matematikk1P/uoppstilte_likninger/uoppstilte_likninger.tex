\documentclass[a4, 11pt, twoside]{article}

\usepackage{pstricks,pst-plot}
\usepackage[utf8]{inputenc}
\usepackage[english,norsk]{babel}

\usepackage{listings}
\usepackage{color}
\usepackage{xcolor}

\usepackage{natbib}


\lstset{ %
  backgroundcolor=\color{white},   % choose the background color; you must add \usepackage{color} or \usepackage{xcolor}
  basicstyle=\footnotesize,        % the size of the fonts that are used for the code
  breakatwhitespace=false,         % sets if automatic breaks should only happen at whitespace
  breaklines=true,                 % sets automatic line breaking
  %captionpos=b,                   % sets the caption-position to bottom
  commentstyle=\color{blue},       % comment style
  deletekeywords={...},            % if you want to delete keywords from the given language
  escapeinside={\%*}{*)},          % if you want to add LaTeX within your code
  extendedchars=true,              % lets you use non-ASCII characters; for 8-bits encodings only, does not work with UTF-8
  keepspaces=true,                 % keeps spaces in text, useful for keeping indentation of code (possibly needs columns=flexible)
  keywordstyle=\color{orange},       % keyword style
  language=Python,                 % the language of the code
  morekeywords={*,...},            % if you want to add more keywords to the set
  rulecolor=\color{black},         % if not set, the frame-color may be changed on line-breaks within not-black text (e.g. comments (green here))
  showspaces=false,                % show spaces everywhere adding particular underscores; it overrides 'showstringspaces'
  showstringspaces=false,          % underline spaces within strings only
  showtabs=false,                  % show tabs within strings adding particular underscores
  stepnumber=2,                    % the step between two line-numbers. If it's 1, each line will be numbered
  stringstyle=\color{violet},      % string literal style
  tabsize=2,                       % sets default tabsize to 2 spaces
}

\usepackage{caption}
\DeclareCaptionFont{white}{\color{white}}
\DeclareCaptionFormat{listing}{\colorbox{gray}{\parbox{\textwidth}{#1#2#3}}}
\captionsetup[lstlisting]{format=listing,labelfont=white,textfont=white}

% This concludes the preamble

\usepackage[colorlinks=true,citecolor=red]{hyperref}

% \usepackage[norsk]{babel}
\usepackage{amsmath, amssymb, amsthm}
\usepackage{makeidx}
\usepackage{graphicx}

\usepackage{fancyhdr}
\pagestyle{fancy}

\usepackage{accents}
\newcommand{\interior}[1]{\accentset{\smash{\raisebox{-0.12ex}{$\scriptstyle\circ$}}}{#1}\rule{0pt}{2.3ex}}
\fboxrule0.0001pt \fboxsep0pt

% used for diagrams, not needed here
% \usepackage{tikz}
% \usepackage{tikz-cd}
% \usetikzlibrary{matrix, arrows, decorations}



%%% New environments
\newtheorem{theorem}{Theorem}[section]
\newtheorem{lemma}[theorem]{Lemma}
\newtheorem{corollary}[theorem]{Corollary}
\newtheorem{prop}[theorem]{Proposition}

\theoremstyle{definition}
\newtheorem{defn}[theorem]{Definition}
\newtheorem{example}[theorem]{Example}
\newtheorem{eksempel}[theorem]{Eksempel}
\newtheorem{exercise}[theorem]{Exercise}
\newtheorem{remark}[theorem]{Remark}
\newtheorem{nremark}[theorem]{Bemerkning}
\newtheorem{question}[theorem]{Question}
\newtheorem{conjecture}[theorem]{Conjecture}
\newtheorem{improvement}[theorem]{Improvement}
\newtheorem{discus}[theorem]{Discus}
\newtheorem{ptheorem}[theorem]{Possible Theorem}
\newtheorem{project}[theorem]{Project}
\newtheorem{solution}[theorem]{Solution}
\newcommand\hra{\hookrightarrow}

%%% Custom definitions and macros


\DeclareMathOperator{\vspan}{Span}
\renewcommand{\d}{\mathrm{\; d}} % for differensialet i integraler.

\newcommand{\Aut}{\mathop{{\rm Aut}}}
\newcommand{\End}{\mathop{{\rm End}}}
\newcommand{\Hom}{\mathop{{\rm Hom}}}
\newcommand{\rank}{\mathop{{\rm rank}}}
\newcommand{\st}{\text{ s.t }}
\renewcommand{\div}{\mathop{{\rm div}}}

\DeclareMathOperator{\Dx}{\frac{\d}{\d x}}
\DeclareMathOperator{\DDx}{\frac{\d^2}{\d x^2}}
\DeclareMathOperator{\erf}{erf}
\DeclareMathOperator{\erfc}{erfc}
\DeclareMathOperator{\sign}{sgn}


% Caligraphic letters
\newcommand{\cA}{\mathcal{A}}
\newcommand{\cB}{\mathcal{B}}
\newcommand{\cR}{\mathcal{R}}
\newcommand{\cC}{\mathcal{C}}
\newcommand{\cD}{\mathcal{D}}
\newcommand{\cE}{\mathcal{E}}
\newcommand{\cF}{\mathcal{F}}
\newcommand{\cG}{\mathcal{G}}
\newcommand{\cH}{\mathcal{H}}
\newcommand{\cI}{\mathcal{I}}
\newcommand{\cJ}{\mathcal{J}}
\newcommand{\cK}{\mathcal{K}}
\newcommand{\cL}{\mathcal{L}}
\newcommand{\cN}{\mathcal{N}}
\newcommand{\cO}{\mathcal{O}}
\newcommand{\cS}{\mathcal{S}}
\newcommand{\cZ}{\mathcal{Z}}
\newcommand{\cP}{\mathcal{P}}
\newcommand{\cT}{\mathcal{T}}
\newcommand{\cU}{\mathcal{U}}
\newcommand{\cV}{\mathcal{V}}
\newcommand{\cX}{\mathcal{X}}
\newcommand{\cW}{\mathcal{W}}

\newcommand{\A}{\mathbb{A}}
\newcommand{\B}{\mathbb{B}}
\newcommand{\C}{\mathbb{C}}
\newcommand{\D}{\mathbb{D}}
\renewcommand{\H}{\mathbb{H}}
\newcommand{\N}{\mathbb{N}}
\newcommand{\K}{\mathbb{K}}
\newcommand{\Q}{\mathbb{Q}}
\newcommand{\Z}{\mathbb{Z}}
\renewcommand{\P}{\mathbb{P}}
\newcommand{\R}{\mathbb{R}}
\newcommand{\U}{\mathbb{U}}
\newcommand{\bT}{\mathbb{T}}
\newcommand{\bD}{\mathbb{D}}

\newcommand{\vu}{\boldsymbol{u}}
\newcommand{\vv}{\boldsymbol{v}}
\newcommand{\vn}{\boldsymbol{n}}
\newcommand{\vf}{\boldsymbol{f}}

\def\di{\partial}
\def\bs{\backslash}
\def\e{\epsilon}
\def\la{\langle}
\def\ra{\rangle}



\renewcommand{\d}{\;\text{d}}
\newcommand{\ocM}{\overline{\mathcal{M}}^{+}}
\newcommand{\ocB}{\overline{\mathcal{B}}}
\newcommand{\ocA}{\overline{\mathcal{A}}}
\newcommand{\Lp}{\text{L}}
\newcommand{\aev}{\text{ a.e }}
\newcommand{\aeev}{\text{-a.e }}

\newcommand{\llra}{\xrightarrow{\hspace*{1cm}}}

\DeclareMathOperator*{\supp}{supp}
\DeclareMathOperator*{\essup}{essup}


\fancyhead[LE, RO]{Algebra}
\fancyhead[RE, LO]{Uoppstilte likninger}
\setlength{\headheight}{14pt}

\usepackage{exercise}
\def\ExerciseName{Oppgave}


\begin{document}

\section*{Repetisjon/oppvarming}
Her er noen videoer dere kan se før dere begynner på oppgavesettet og
underveis:
Introduksjonsvideo:
\begin{itemize}
  \item \url{https://www.youtube.com/watch?v=la9PFrpt85M}
\end{itemize}
Flere videoer:
\begin{itemize}
  \item \url{https://www.youtube.com/watch?v=cHdzKLuz8LE} (medlemmer i en svømmeklubb)
  \item \url{https://www.youtube.com/watch?v=YYijfTD3WvQ} (differansen til to tall)
\end{itemize}
\begin{Exercise}
Løs likningene med hensyn på $x$ ved regning: \newline
\begin{itemize}
\item[\bf a)] \[ \frac{x}{3} = \frac{2}{3} \]
\item[\bf b)] \[ \frac{x}{5} = \frac{2}{3} \]
\item[\bf c)] \[8(8x - 6) + 14 = 94 \]
\item[\bf d)] \[-11(7x - 2) + 18 = -37\]
\end{itemize}
\end{Exercise}


\section*{Uoppstilte likninger}
Det er viktig at dere leser teksten nøye. Bruk gjerne
følgende tabell til å tolke teksten

\begin{center}
\begin{tabular}{|l | l | l |}
\hline
fem mindre enn $x$ & $x - 5$ \\ \hline 
fem mer enn $x$ &  $x + 5$ \\ \hline
fem ganger så mye som $x$ & $5x$ \\ \hline
halvparten av $x$ & $\frac{x}{2}$ \\ \hline
$x$ sjudeler & $\frac{x}{7}$ \\ \hline
\end{tabular}
\end{center}
\begin{eksempel}
  Vi ser på en typisk oppgave: \newline
Anders har tre ganger så mange penger som
 Kari. Anta at Kari har $x$ kroner.
\begin{itemize}
 \item[\bf a)] Sett opp et utrykk som beskriver hvor mye penger Kari og Anders har til sammen
 \item[\bf b)] Anta at de til sammen har 1000 kr. Hvor mye penger har hver av dem?
\end{itemize}
LØSNING: Siden oppgaven sier at Kari har $x$ kroner, og at Anders har
tre ganger så mange kroner, må altså Anders ha $3\cdot x = 3x$ kroner.
Utrykket som forteller hvor mye de har til sammen, er selvsagt pengene
til Kari og Anders lagt sammen:
\begin{itemize}
  \item[\bf a)] Kari sine penger + Anders sine penger $ = x + 3x = 4x$.
  \item[\bf b)] Til sammen skal de ha 1000kr. Altså må
    \[4x = 1000 \]
    Vi løser likningen og får da
    \begin{align*}
      4x & = 1000 \\
      \frac{4x}{4} & = \frac{1000}{4} \\
      x & = 250
    \end{align*}
\begin{itemize}
\end{eksempel}

\begin{Exercise}
Anders har dobbelt så mange penger som
 Kari. Anta at Kari har $x$ kroner.
\begin{itemize}
 \item[\bf a)] Sett opp et utrykk som beskriver hvor mye penger Kari og Anders har til sammen
 \item[\bf b)] Anta at de til sammen har 1000 kr. Hvor mye penger har hver av dem?
\end{itemize}
\end{Exercise}

\begin{Exercise}
Ove har 200 kroner mer enn
 Per. Anta at Per har $x$ kroner.
\begin{itemize}
 \item[\bf a)] Sett opp et utrykk som beskriver hvor mye penger Ove og Per har til sammen
 \item[\bf b)] Anta at de til sammen har 1200 kr. Hvor mye penger har hver av dem?
\end{itemize}
\end{Exercise}

\begin{Exercise}
Anders har dobbelt så mange penger som
 Anne. Kari har 400 kr mer enn Anne. Anta at Anne har $x$ kroner.
\begin{itemize}
 \item[\bf a)] Sett opp et utrykk som beskriver hvor mye penger Kari, Anders og Anne har til 
 sammen.
 \item[\bf b)] Anta at de til sammen har 5200 kr. Hvor mye penger har hver av dem?
\end{itemize}
\end{Exercise}

\begin{Exercise}
Anders har dobbelt så mange penger som Anne. 
Kari har 1200 kr mer enn Anne.
\begin{itemize}
 \item[\bf a)] Sett opp et utrykk som beskriver hvor mye penger Kari, Anders og Anne har til 
 sammen.
 \item[\bf b)] Anta at de til sammen har 7600 kr. Hvor mye penger har hver av dem?
\end{itemize}
\end{Exercise}

\begin{Exercise}
  Ole er 3 ganger så gammel som Pål, som er fire år eldre enn Jens.
  Til sammen er de 146 år gamle. Hvor gamle er hver av dem?
\end{Exercise}

\begin{Exercise}
I fotball får et lag tre poeng for seier og ett poeng for uavgjort.
I fjor vant laget til Odd ni flere kamper enn de spilte uavgjort. De
fikk til sammen 67 poeng. \newline
Hvor mange kamper vant laget til Odd?
\end{Exercise}

\begin{Exercise}
Jens skal lage en bunnplate til en tank. Bunnplaten skal være
formet som et rektangel.
Rundt platen skal det settes en svært dyr
pakning til 40000 kr/m. For å få en akseptabel pris på
tanken, har han 20m pakning til rådighet på hver plate.
\begin{itemize}
  \item[\bf a)] Forklar at omkretsen til platen må være 20m
\end{itemize}
Vi lar to av sidene i bunnplaten være $x$ meter lange.
\begin{itemize}
  \item[\bf b)] Forklar at de to andre sidene da er $x - 10$ meter
    lange.
  \item[\bf c)] Forklar at arealet av platen da blir gitt ved funksjonen
    \[ A(x) = x(x - 10). \]
\end{itemize}
\end{Exercise}


\begin{Exercise}
  I et spill kastes en blå terning og en hvit terning. Om spilleren
  klarer en utfordring som trekkes fra en kortstokk, får de trekke
  en spesiell regel fra en regel-kortstokk, som forteller
  hvor mange steg spilleren får fortsette etter å ha kastet
  terningene.\newline
  Åse klarer sin utfordring, og får trekke et regel-kort.
      På kortet står det: ''Gang antall øyne på blå terning med 5,
      og trekk fra 2 ganger antall øyne på hvit terning''
  \begin{itemize}
    \item[\bf a)] Hvis Åse får 3 blå øyne og 6 hvite øyne, hvor mange
      ganger kan hun flytte brikken sin?
    \item[\bf b)] Sett opp et bokstavuttrykk eller en formel som
      forklarer hvor mange steg Åse kan ta med denne regelen.
  \end{itemize}
  På et annet kort står det: \newline
    ''Antall steg du får ta, ganget med antall blå øyne, er 
    19 minus antall hvite øyne.''
  \begin{itemize}
    \item[\bf c)] Hvis antall blå og hvite øyne var som i oppgave a),
      finn likningen som beskriver antall steg man kan ta.
    \item[\bf d)] Løs likningen du fant i c)
    \item[\bf e)] Hva er den generelle likningen som fungerer for alle
      terningkastene?
  \end{itemize}
\end{Exercise}


\begin{Exercise}
\begin{itemize}
 \item[\bf a)]
 Britt, Ali og Ole er ute og reiser.
 Britt og Ali bruker like mye tid. Ole bruker 30 timer mer enn de to andre.
 Til sammen reiser de i 120 timer. Still opp en likning, og regn ut
 hvor lenge hver av dem reiser.

 \item[\bf b)]
 En gruppe på fire personer bestående av Åse, Bob, Janne og Fredrik skal
 ut å reise. Til sammen trenger de
 $8,9343\cdot10^4\text{kr}$.
 \begin{itemize}
  \item Åse betaler 3,5 ganger så mye som Kåre
  \item Bob betaler 244 kroner mer enn Åse
  \item Janne betaler halvparten så mye som Kåre
  \item Fredrik betaler $\frac{7}{2}$ ganger så mye som Janne.
 \end{itemize}
 Sett opp en likning som beskriver situasjonen. Forklar hvilken
 person hvert ledd representerer, og finn ut hvor mye penger hver
 av dem betaler.
\end{itemize}
\end{Exercise}

\begin{Exercise}
Formelen for BMI er gitt ved \[I = \frac{m}{h^2}, \]
der $I$  er BMI (body mass index), $m$ er vekten til personen målt i kilogram
($kg$),
    og $h$  er høyden målt i meter. 
    Kåre veier $80 kg$ og har en BMI på $20$. Snu formelen for å få en
    formel for høyden, og regn ut hvor høy Kåre er.
\end{Exercise}

\begin{Exercise}
Vi ser på Ohms lov, \[ U = RI. \]
\begin{itemize}
\item[\bf a)] Anta $R = 5$, og $I = 0,25$. Finn ut hva $U$ er.
\item[\bf b)] Anta $U = 12$, og at $R = 36$. Sett tallene inn i Ohms lov, og vis at du får en likning
for $I$. Løs denne, og finn ut hva $I$ må være.
\item[\bf c)] Anta $U = 12$, og $I = 0.5$. Sett tallene inn i Ohms lov, og vis at du får en likning for
$R$. Løs denne likningen, og finn ut hva $R$ må være.
\item[\bf d)] Snu Ohms lov, og finn en formel for $I$. Sett inn tallene for $U$ og $R$ fra oppgave b),
og sammenlign svarene.
\item[\bf e)] Snu Ohms lov, og finn en formel for $R$. Sett inn tallene for $U$ og $I$ fra oppgave c),
og sammenlign svarene.
\end{itemize}
\end{Exercise}

\end{document}
