\documentclass[a4, 11pt, twoside]{article}

\usepackage{pstricks,pst-plot}
\usepackage[utf8]{inputenc}
\usepackage[english,norsk]{babel}

\usepackage{ulem}

\usepackage{listings}
\usepackage{color}
\usepackage{xcolor}

\usepackage{natbib}


\lstset{ %
  backgroundcolor=\color{white},   % choose the background color; you must add \usepackage{color} or \usepackage{xcolor}
  basicstyle=\footnotesize,        % the size of the fonts that are used for the code
  breakatwhitespace=false,         % sets if automatic breaks should only happen at whitespace
  breaklines=true,                 % sets automatic line breaking
  %captionpos=b,                   % sets the caption-position to bottom
  commentstyle=\color{blue},       % comment style
  deletekeywords={...},            % if you want to delete keywords from the given language
  escapeinside={\%*}{*)},          % if you want to add LaTeX within your code
  extendedchars=true,              % lets you use non-ASCII characters; for 8-bits encodings only, does not work with UTF-8
  keepspaces=true,                 % keeps spaces in text, useful for keeping indentation of code (possibly needs columns=flexible)
  keywordstyle=\color{orange},       % keyword style
  language=Python,                 % the language of the code
  morekeywords={*,...},            % if you want to add more keywords to the set
  rulecolor=\color{black},         % if not set, the frame-color may be changed on line-breaks within not-black text (e.g. comments (green here))
  showspaces=false,                % show spaces everywhere adding particular underscores; it overrides 'showstringspaces'
  showstringspaces=false,          % underline spaces within strings only
  showtabs=false,                  % show tabs within strings adding particular underscores
  stepnumber=2,                    % the step between two line-numbers. If it's 1, each line will be numbered
  stringstyle=\color{violet},      % string literal style
  tabsize=2,                       % sets default tabsize to 2 spaces
}

\usepackage{caption}
\DeclareCaptionFont{white}{\color{white}}
\DeclareCaptionFormat{listing}{\colorbox{gray}{\parbox{\textwidth}{#1#2#3}}}
\captionsetup[lstlisting]{format=listing,labelfont=white,textfont=white}

% This concludes the preamble

\usepackage[colorlinks=true,citecolor=red]{hyperref}

% \usepackage[norsk]{babel}
\usepackage{amsmath, amssymb, amsthm}
\usepackage{makeidx}
\usepackage{graphicx}

\usepackage{fancyhdr}
\pagestyle{fancy}

\usepackage{accents}
\newcommand{\interior}[1]{\accentset{\smash{\raisebox{-0.12ex}{$\scriptstyle\circ$}}}{#1}\rule{0pt}{2.3ex}}
\fboxrule0.0001pt \fboxsep0pt

% used for diagrams, not needed here
% \usepackage{tikz}
% \usepackage{tikz-cd}
% \usetikzlibrary{matrix, arrows, decorations}



%%% New environments
\newtheorem{theorem}{Theorem}[section]
\newtheorem{lemma}[theorem]{Lemma}
\newtheorem{corollary}[theorem]{Corollary}
\newtheorem{prop}[theorem]{Proposition}

\theoremstyle{definition}
\newtheorem{defn}[theorem]{Definition}
\newtheorem{example}[theorem]{Example}
\newtheorem{eksempel}[theorem]{Eksempel}
\newtheorem{exercise}[theorem]{Exercise}
\newtheorem{remark}[theorem]{Remark}
\newtheorem{nremark}[theorem]{Bemerkning}
\newtheorem{advarsel}[theorem]{ADVARSEL}
\newtheorem{question}[theorem]{Question}
\newtheorem{conjecture}[theorem]{Conjecture}
\newtheorem{improvement}[theorem]{Improvement}
\newtheorem{discus}[theorem]{Discus}
\newtheorem{ptheorem}[theorem]{Possible Theorem}
\newtheorem{project}[theorem]{Project}
\newtheorem{solution}[theorem]{Solution}
\newcommand\hra{\hookrightarrow}

%%% Custom definitions and macros


\DeclareMathOperator{\vspan}{Span}
\renewcommand{\d}{\mathrm{\; d}} % for differensialet i integraler.

\newcommand{\Aut}{\mathop{{\rm Aut}}}
\newcommand{\End}{\mathop{{\rm End}}}
\newcommand{\Hom}{\mathop{{\rm Hom}}}
\newcommand{\rank}{\mathop{{\rm rank}}}
\newcommand{\st}{\text{ s.t }}
\renewcommand{\div}{\mathop{{\rm div}}}

\DeclareMathOperator{\Dx}{\frac{\d}{\d x}}
\DeclareMathOperator{\DDx}{\frac{\d^2}{\d x^2}}
\DeclareMathOperator{\erf}{erf}
\DeclareMathOperator{\erfc}{erfc}
\DeclareMathOperator{\sign}{sgn}


% Caligraphic letters
\newcommand{\cA}{\mathcal{A}}
\newcommand{\cB}{\mathcal{B}}
\newcommand{\cR}{\mathcal{R}}
\newcommand{\cC}{\mathcal{C}}
\newcommand{\cD}{\mathcal{D}}
\newcommand{\cE}{\mathcal{E}}
\newcommand{\cF}{\mathcal{F}}
\newcommand{\cG}{\mathcal{G}}
\newcommand{\cH}{\mathcal{H}}
\newcommand{\cI}{\mathcal{I}}
\newcommand{\cJ}{\mathcal{J}}
\newcommand{\cK}{\mathcal{K}}
\newcommand{\cL}{\mathcal{L}}
\newcommand{\cN}{\mathcal{N}}
\newcommand{\cO}{\mathcal{O}}
\newcommand{\cS}{\mathcal{S}}
\newcommand{\cZ}{\mathcal{Z}}
\newcommand{\cP}{\mathcal{P}}
\newcommand{\cT}{\mathcal{T}}
\newcommand{\cU}{\mathcal{U}}
\newcommand{\cV}{\mathcal{V}}
\newcommand{\cX}{\mathcal{X}}
\newcommand{\cW}{\mathcal{W}}

\newcommand{\A}{\mathbb{A}}
\newcommand{\B}{\mathbb{B}}
\newcommand{\C}{\mathbb{C}}
\newcommand{\D}{\mathbb{D}}
\renewcommand{\H}{\mathbb{H}}
\newcommand{\N}{\mathbb{N}}
\newcommand{\K}{\mathbb{K}}
\newcommand{\Q}{\mathbb{Q}}
\newcommand{\Z}{\mathbb{Z}}
\renewcommand{\P}{\mathbb{P}}
\newcommand{\R}{\mathbb{R}}
\newcommand{\U}{\mathbb{U}}
\newcommand{\bT}{\mathbb{T}}
\newcommand{\bD}{\mathbb{D}}

\newcommand{\vu}{\boldsymbol{u}}
\newcommand{\vv}{\boldsymbol{v}}
\newcommand{\vn}{\boldsymbol{n}}
\newcommand{\vf}{\boldsymbol{f}}

\def\di{\partial}
\def\bs{\backslash}
\def\e{\epsilon}
\def\la{\langle}
\def\ra{\rangle}



\renewcommand{\d}{\;\text{d}}
\newcommand{\ocM}{\overline{\mathcal{M}}^{+}}
\newcommand{\ocB}{\overline{\mathcal{B}}}
\newcommand{\ocA}{\overline{\mathcal{A}}}
\newcommand{\Lp}{\text{L}}
\newcommand{\aev}{\text{ a.e }}
\newcommand{\aeev}{\text{-a.e }}

\newcommand{\llra}{\xrightarrow{\hspace*{1cm}}}

\DeclareMathOperator*{\supp}{supp}
\DeclareMathOperator*{\essup}{essup}


\fancyhead[LE, RO]{Tall og Algebra}
\fancyhead[RE, LO]{Veien om 1}
\setlength{\headheight}{14pt}

\usepackage{exercise}
\def\ExerciseName{Oppgave}


\begin{document}

\section*{Repetisjon/oppvarming: brøkregning}
Husk at vi kan forkorte og utvide brøker. Vi kan
gange teller og nevner med samme tall, uten å forandre verdien av brøken.
\begin{eksempel}
\[ \frac{5}{2} = \frac{5\cdot3}{2\cdot3} = \frac{15}{6}\]
Dette vanligvis å utvide brøken.
På akkurat samme måte kan vi forkorte brøker. I eksempelet over har vi da
at
\[\frac{15}{6} = \frac{5\cdot3}{2\cdot3} = \frac{5}{2}\]
Dette blir ofte kalt å forkorte brøken, eller å stryke felles faktorer.
\end{eksempel}

Her er noen videoer dere kan se før dere begynner på oppgavesettet og
underveis:
Introduksjonsvideoer:
\begin{itemize}
  \item \url{https://www.youtube.com/watch?v=eBKho1HIWT4}
  \item \url{https://www.youtube.com/watch?v=vXlNh9UHmXc}
\end{itemize}

\begin{Exercise}
Skriv som prosent og desimaltall:
    \begin{center}
    \begin{tabular}{l l l l l l}
        \textbf{a)} $\displaystyle \frac{1}{4}$ &
        \textbf{b)} $\displaystyle \frac{3}{4}$ &
        \textbf{c)} $\displaystyle \frac{1}{3}$ &
        \textbf{d)} $\displaystyle \frac{2}{3}$ &
        \textbf{e)} $\displaystyle \frac{2}{5}$ &
        \textbf{e)} $\displaystyle \frac{7}{10}$ &
    \end{tabular}
    \end{center}
\end{Exercise}

\begin{Exercise}
Skriv som prosent og desimaltall:
    \begin{center}
    \begin{tabular}{l l l l l l}
        \textbf{a)} $\displaystyle \frac{7}{11}$ &
        \textbf{b)} $\displaystyle \frac{3}{7}$ &
        \textbf{c)} $\displaystyle \frac{8}{13}$ &
        \textbf{d)} $\displaystyle \frac{14}{19}$ &
        \textbf{e)} $\displaystyle \frac{17}{23}$ &
        \textbf{e)} $\displaystyle \frac{8}{14}$ &
    \end{tabular}
    \end{center}
\end{Exercise}

\begin{Exercise}
Per liker å trene. Han betaler 450kr i måneden for et treningsstudio.
I en vanlig måned trener han 25 ganger. Hvor mye betaler han per
trening i en vanlig måned?
\end{Exercise}

\begin{Exercise}
Jens gikk en tur på 2,5 km på 20 min.
\begin{itemize}
\item[\bf a)] Hvor mange minutter brukte Jens på én km?
\item[\bf b)] Hvis han holder samme hastighet,
hvor lang tid bruker han på å gå 3,4km ?
\end{itemize}
\end{Exercise}

\begin{Exercise}
En iskrem kommer i to forskjellige innpakninger.
En koster 24,90kr for 240g, mens en annen koster 
27,70 for 330g. Hvilken pakke lønner det seg å kjøpe?
\end{Exercise}

\begin{Exercise}
En pakke med torsk på 675g fra produsenten ''Fuskeoppdrett AS''
koster 114kr, mens en pakke på 520g fra produsenten ''Sjarkfisk''
koster 90kr. Hvilken produsent har billigst fisk?
\end{Exercise}

\begin{Exercise}
  I en kakeoppskrift trengs det 750g hvetemel til 6 egg.
  Du har bare 2 egg på en søndag, men veldig lyst på kake.
  Hvor mange gram hvetemel trengs det til en oppskrift på 2 egg?
\end{Exercise}

\begin{Exercise}
Din lokale klesforretning har en bukse på tilbud. Ordinær pris er
420kr, og buksen er satt ned 25\%.
\begin{itemize}
\item[\bf a)] Hvor mye er én prosent av ordinær pris?
\item[\bf b)] Hvor mange kroner utgjorde rabatten?
\item[\bf c)] Hva var tilbudsprisen?
\end{itemize}
\end{Exercise}



\begin{Exercise}
Din lokale klesforretning har en skjorte på tilbud. Ordinær pris er
540kr, og skjorta er satt ned 20\%.
\begin{itemize}
\item[\bf a)] Hvor mye er én prosent av ordinær pris?
\item[\bf b)] Hvor mange kroner utgjorde rabatten?
\item[\bf c)] Hva var tilbudsprisen?
\end{itemize}
\end{Exercise}

\begin{Exercise}
Din lokale klesforretning har en skjorte på tilbud. Tilbudspris er
430kr, og skjorta er satt ned 20\% (av ordinær pris).
\begin{itemize}
\item[\bf a)] Forklar at tilbudsprisen utgjør 80\% av ordinær pris
\item[\bf b)] Hvor mye er én prosent av ordinær pris? (gå veien om 1)
\item[\bf c)] Hva var ordinær pris?
\item[\bf d)] Hvor mange kroner utgjorde rabatten?
\end{itemize}
\end{Exercise}


\begin{Exercise}
    Per satte penger på sparekonto med 5\% rente 1. januar 2015.
    1. januar 2016 sto det 5600 kroner på kontoen. Vi antar at
    kontoen var tom før Per satte inn penger 1. januar 2016 og at
    han ikke har rørt kontoen siden. \newline 
    Hvor mange kroner satte Per inn på kontoen 1. januar 2015 ?
\end{Exercise}

\begin{Exercise}
    Et selskap kjøpte brukt utstyr for 7 300 000 kroner. Vi antar
    at utstyret har sunket med 19\% siden det var nytt. 
    \newline
    Hva kostet utstyret når det var nytt ?
\end{Exercise}

\begin{Exercise}
    Jens skal kjøpe moped. Han har hørt at verdien på nye mopeder synker med
    12\% årlig. Han ser en ny moped til 34000 kroner, og har tilbud om en ett år gammel 
    moped av samme modell til 31000 kroner.
    \begin{description}
        \item[a)] Hvilken moped ville du valgt ?
        \item[b)] Hva kostet mopeden til 31000 kroner som ny dersom prisen på den har sunket 12\% siden i fjor ?
    \end{description}
\end{Exercise}

\begin{Exercise}
Per skal gå en tur på fjellet. Han har trent i lang tid, men på noe
kortere turer. Han går 20km på 4 timer under treningsøktene.
\begin{itemize}
\item[\bf a)] Hvor lang tid bruker han på 35km med denne farta?
\item[\bf b)] Per regner med å gå 7 timer hver dag mens han er i fjellet.
    Han regner også med å holde 80\% av farten når han trente. Hvor langt
        går han hver dag ?
\end{itemize}
\end{Exercise}

\begin{Exercise}
Petter skal male en vegg. Han bruker 5L maling på 7m$^2$. Hvor mye
    maling bruker han på vegg med areal $50\text{m}^2$ ?
\end{Exercise}

\begin{Exercise}
Gjør oppgavene 154, 156, og 157 side 238 i læreboken ''Matematikk 1P, Aschehoug ,2009''.
\end{Exercise}

\section*{Mer avanserte teknikker med veien om 1 (forhold,
saftblandinger, finne gammel verdi i prosentregning)}
Her kommer det flere oppgaver senere.

\end{document}
