\documentclass[a4, 11pt, twoside]{article}

%\usepackage{pstricks,pst-plot}
\usepackage[utf8]{inputenc}
\usepackage[english,norsk]{babel}

\usepackage{pgf,tikz}
\usepackage{tkz-euclide}
\usepackage{mathrsfs}


\usetikzlibrary{arrows, intersections}
\tikzset{
  hpil/.style={
          ->,
          thick,
          shorten <=2pt,
          shorten =>2pt,}
}
\tikzset{
  vpil/.style={
          <-,
          thick,
          shorten =>2pt,}
}
\newcommand{\degre}{\ensuremath{^\circ}}

\usepackage{ulem}

\usepackage{listings}
\usepackage{color}
\usepackage{xcolor}

% ------------------------------ package tcolorbox with definitions ----------------------
\usepackage{tcolorbox}
\tcbset{textmarker/.style={%
    skin=enhancedmiddle jigsaw, breakable, parbox=false,
    boxrule=0mm, leftrule=0mm, rightrule=0mm, boxsep=0mm, arc=0mm, outer arc=0mm,
    left=3mm, right=3mm, top=2mm, toptitle=2mm, bottomtitle=1mm, oversize }}

    \newtcolorbox{yellow}{textmarker, colback=yellow!45!white, colframe=yellow}
    \newtcolorbox{blue}{textmarker, colback=blue!20!white, colframe=blue}
    \newtcolorbox{green}{textmarker, colback=green!30!white, colframe=green}


% ----------------------------------------------------------------------------------------

\usepackage{natbib}


\lstset{ %
  backgroundcolor=\color{white},   % choose the background color; you must add \usepackage{color} or \usepackage{xcolor}
  basicstyle=\footnotesize,        % the size of the fonts that are used for the code
  breakatwhitespace=false,         % sets if automatic breaks should only happen at whitespace
  breaklines=true,                 % sets automatic line breaking
  %captionpos=b,                   % sets the caption-position to bottom
  commentstyle=\color{blue},       % comment style
  deletekeywords={...},            % if you want to delete keywords from the given language
  escapeinside={\%*}{*)},          % if you want to add LaTeX within your code
  extendedchars=true,              % lets you use non-ASCII characters; for 8-bits encodings only, does not work with UTF-8
  keepspaces=true,                 % keeps spaces in text, useful for keeping indentation of code (possibly needs columns=flexible)
  keywordstyle=\color{orange},       % keyword style
  language=Python,                 % the language of the code
  morekeywords={*,...},            % if you want to add more keywords to the set
  rulecolor=\color{black},         % if not set, the frame-color may be changed on line-breaks within not-black text (e.g. comments (green here))
  showspaces=false,                % show spaces everywhere adding particular underscores; it overrides 'showstringspaces'
  showstringspaces=false,          % underline spaces within strings only
  showtabs=false,                  % show tabs within strings adding particular underscores
  stepnumber=2,                    % the step between two line-numbers. If it's 1, each line will be numbered
  stringstyle=\color{violet},      % string literal style
  tabsize=2,                       % sets default tabsize to 2 spaces
}

\usepackage{caption}
\DeclareCaptionFont{white}{\color{white}}
\DeclareCaptionFormat{listing}{\colorbox{gray}{\parbox{\textwidth}{#1#2#3}}}
\captionsetup[lstlisting]{format=listing,labelfont=white,textfont=white}

% This concludes the preamble

\usepackage[colorlinks=true,citecolor=red]{hyperref}

% \usepackage[norsk]{babel}
\usepackage{amsmath, amssymb, amsthm}
\usepackage{makeidx}
\usepackage{graphicx}

\usepackage{fancyhdr}
\pagestyle{fancy}

\usepackage{accents}
\newcommand{\interior}[1]{\accentset{\smash{\raisebox{-0.12ex}{$\scriptstyle\circ$}}}{#1}\rule{0pt}{2.3ex}}
\fboxrule0.0001pt \fboxsep0pt

% used for diagrams, not needed here
% \usepackage{tikz}
% \usepackage{tikz-cd}
% \usetikzlibrary{matrix, arrows, decorations}

%\newcommand\xqed[1]{%
%  \leavevmode\unskip\penalty9999 \hbox{}\nobreak\hfill
%  \quad\hbox{#1}}
%\newcommand\demo{\xqed{$\triangle$}}


\definecolor{shadecolor}{RGB}{150,150,150}
\newcommand{\examplebox}[1]{\par\noindent\colorbox{shadecolor}
{\parbox{\dimexpr\textwidth-2\fboxsep\relax}{#1}}}


%%% New environments
\newtheorem{theorem}{Theorem}[section]
\newtheorem{ntheorem}{Teorem}[section]
\newtheorem{lemma}[theorem]{Lemma}
\newtheorem{corollary}[theorem]{Corollary}
\newtheorem{prop}[theorem]{Proposition}

\theoremstyle{definition}
\newtheorem{defn}[theorem]{Definition}
\newtheorem{example}[theorem]{Example}
\newtheorem{eksempelx}{Eksempel}
\newenvironment{eksempel}
   {\pushQED{\qed}\renewcommand{\qedsymbol}{%\textbf{eksempel slutt:}
   $\clubsuit$}\eksempelx}
   {\popQED\endeksempelx}
\setcounter{eksempel}{0}
\newtheorem{exercise}[theorem]{Exercise}
\newtheorem{remark}[theorem]{Remark}
\newtheorem{nremark}[theorem]{Bemerkning}
\newtheorem{advarsel}[theorem]{ADVARSEL}
\newtheorem{question}[theorem]{Question}
\newtheorem{conjecture}[theorem]{Conjecture}
\newtheorem{improvement}[theorem]{Improvement}
\newtheorem{discus}[theorem]{Discus}
\newtheorem{ptheorem}[theorem]{Possible Theorem}
\newtheorem{project}[theorem]{Project}
\newtheorem{solution}[theorem]{Solution}
\newcommand\hra{\hookrightarrow}



%%% Custom definitions and macros


\DeclareMathOperator{\vspan}{Span}
\renewcommand{\d}{\mathrm{\; d}} % for differensialet i integraler.

\newcommand{\Aut}{\mathop{{\rm Aut}}}
\newcommand{\End}{\mathop{{\rm End}}}
\newcommand{\Hom}{\mathop{{\rm Hom}}}
\newcommand{\rank}{\mathop{{\rm rank}}}
\newcommand{\st}{\text{ s.t }}
\renewcommand{\div}{\mathop{{\rm div}}}

\DeclareMathOperator{\Dx}{\frac{\d}{\d x}}
\DeclareMathOperator{\DDx}{\frac{\d^2}{\d x^2}}
\DeclareMathOperator{\erf}{erf}
\DeclareMathOperator{\erfc}{erfc}
\DeclareMathOperator{\sign}{sgn}


% Caligraphic letters
\newcommand{\cA}{\mathcal{A}}
\newcommand{\cB}{\mathcal{B}}
\newcommand{\cR}{\mathcal{R}}
\newcommand{\cC}{\mathcal{C}}
\newcommand{\cD}{\mathcal{D}}
\newcommand{\cE}{\mathcal{E}}
\newcommand{\cF}{\mathcal{F}}
\newcommand{\cG}{\mathcal{G}}
\newcommand{\cH}{\mathcal{H}}
\newcommand{\cI}{\mathcal{I}}
\newcommand{\cJ}{\mathcal{J}}
\newcommand{\cK}{\mathcal{K}}
\newcommand{\cL}{\mathcal{L}}
\newcommand{\cN}{\mathcal{N}}
\newcommand{\cO}{\mathcal{O}}
\newcommand{\cS}{\mathcal{S}}
\newcommand{\cZ}{\mathcal{Z}}
\newcommand{\cP}{\mathcal{P}}
\newcommand{\cT}{\mathcal{T}}
\newcommand{\cU}{\mathcal{U}}
\newcommand{\cV}{\mathcal{V}}
\newcommand{\cX}{\mathcal{X}}
\newcommand{\cW}{\mathcal{W}}

\newcommand{\A}{\mathbb{A}}
\newcommand{\B}{\mathbb{B}}
\newcommand{\C}{\mathbb{C}}
\newcommand{\D}{\mathbb{D}}
\renewcommand{\H}{\mathbb{H}}
\newcommand{\N}{\mathbb{N}}
\newcommand{\K}{\mathbb{K}}
\newcommand{\Q}{\mathbb{Q}}
\newcommand{\Z}{\mathbb{Z}}
\renewcommand{\P}{\mathbb{P}}
\newcommand{\R}{\mathbb{R}}
\newcommand{\U}{\mathbb{U}}
\newcommand{\bT}{\mathbb{T}}
\newcommand{\bD}{\mathbb{D}}

\newcommand{\vu}{\boldsymbol{u}}
\newcommand{\vv}{\boldsymbol{v}}
\newcommand{\vn}{\boldsymbol{n}}
\newcommand{\vf}{\boldsymbol{f}}

\def\di{\partial}
\def\bs{\backslash}
\def\e{\epsilon}
\def\la{\langle}
\def\ra{\rangle}



\renewcommand{\d}{\;\text{d}}
\newcommand{\ocM}{\overline{\mathcal{M}}^{+}}
\newcommand{\ocB}{\overline{\mathcal{B}}}
\newcommand{\ocA}{\overline{\mathcal{A}}}
\newcommand{\Lp}{\text{L}}
\newcommand{\aev}{\text{ a.e }}
\newcommand{\aeev}{\text{-a.e }}
\newcommand{\cm}{\text{cm}}
\newcommand{\m}{\text{m}}
\newcommand{\mm}{\text{mm}}
\newcommand{\km}{\text{km}}

\newcommand{\llra}{\xrightarrow{\hspace*{1cm}}}

\DeclareMathOperator*{\supp}{supp}
\DeclareMathOperator*{\essup}{essup}


\fancyhead[LE, RO]{Ukeoppgaver Uke34}
\fancyhead[RE, LO]{22 - 29 August}
\setlength{\headheight}{14pt}


%\usepackage[noanswer]{exercise}
\usepackage{exercise}
\def\ExerciseName{Oppgave}
\begin{document}
Kort om oppgavene i læreboka: de er nummerert etter kapittel(første tall), og
oppgavenummer innen kapittelet. Oppgave 1.2 er altså andre oppgave i første kapittel.

I tilleg er det en oppgavesamling bakerst i boken, som er nummerert med kapittelet på første siffer,
og stigende rekkefølge innen kapittelet. 100 er første oppgave for kapittel 1, 201 er andre oppgave for kapittel 2 osv.
\section{Negative tall}
\subsection{Oppgaver i læreboka}
Oppgavene 1.1 - 1.7
Oppgaver etter hvor mye måloppnåelse dere ønsker å oppnå:

\begin{center}
\begin{tabular}{| l | l | l | l |}
\hline
\textbf{Lav (sti 1)}         & \textbf{Middels (sti 2)}       &       \textbf{Høy (sti 3)}      \\ \hline
100, 102, 103           & 100, 102, 105                       & 100, 102, 105, 106, 107, 109    \\ \hline
\end{tabular}
\begin{center}

Det skader selvfølgelig ikke å gjøre flere!

\subsection{Flere oppgaver}

Vi kan også bruke tallinja for å få et visuelt bilde av hva vi gjør:
La oss for eksempel legge sammen:
\begin{eksempel}
Vi skal regne ut 2 + 3:
\begin{center}
\begin{tikzpicture}[line cap=round,line join=round, >=angle 90 ,x=0.9cm,y=0.9cm]
\draw[->] (-4.3,0.) -- (8.7,0.);
\foreach \x in {-4,-3,-2,-1, 0, 1,2,3,4,5,6,7,8}
\draw[shift={(\x,0)},color=black] (0pt,2pt) -- (0pt,-2pt) node[below] {\footnotesize $\x$};
\draw (8.8, 0) node[above] {$\Z$};

\draw[hpil] (2, .1) to [out=30, in=150] (5, .1);
\clip(-4.3,-0.5) rectangle (9.5,0.5);
\end{tikzpicture}
\end{center}
Vi kan visualisere dette med å starte på tallet 2, og telle oss 3 skritt fremover til
tallet 5.
\end{eksempel}

\begin{Exercise}
Regn ut ved å bruke en tallinje:
\begin{center}
\begin{tabular}{l l l l l}
& \textbf{a)} $2 + 7$
& \textbf{b)} $-4 + 4$
& \textbf{c)} $-5 + 5$
& \textbf{d)} $-3 + 6$
\end{tabular}
\end{center}
\end{Exercise}

\begin{Exercise}
Pål og Anders skal kjøpe godteri. Til sammen har de 110 kr. 
Pål har 75 kroner. Hvor mange kroner har Anders ?
\end{Exercise}

\begin{Exercise}
  Anne skal handle disse varene i butikken, og har anslått prisen på hver varekategori.
  \begin{itemize}
    \item Brus til 72 kroner
    \item Kjeks til 25 kroner
    \item Is til 36 kroner
  \end{itemize}
  Hvor mye regner Anne med å betale til sammen ?
\end{Exercise}

\begin{Exercise}
Regn ut:
\begin{center}
\begin{tabular}{l l l l l}
& \textbf{a)} $(-3)\cdot(-5)$
& \textbf{b)} $(-3)\cdot(5)$
& \textbf{c)} $5\cdot(-2)\cdot(-4)$
& \textbf{d)} $(-3)\cdot(-3)\cdot(-3)$
\end{tabular}
\end{center}
\end{Exercise}

\begin{Exercise}\label{skriv_som_potens}
Skriv som potens:
\begin{center}
\begin{tabular}{l l l l l}
& \textbf{a)} $3\cdot 3$
& \textbf{b)} $(-3)\cdot(-3)$
& \textbf{c)} $4\cdot 4\cdot 4$
& \textbf{d)} $(-4)\cdot(-4)\cdot(-4)$
\end{tabular}
\end{center}
\end{Exercise}

\begin{Exercise}\label{regn_ut_potenser}
Regn ut svarene på regnestykkene i oppgave \ref{skriv_som_potens}
\end{Exercise}

\begin{Exercise}
Potenser på kalkulator: For å regne ut f.eks $3^5$ på kalkulatoren vår, bruker vi hatte-tasten
$\verb ^ $ ved å taste 3\verb ^ 5 = $3^5 = 243$.
\newline
Sjekk svarene dine i oppgave \ref{regn_ut_potenser} med kalkulator
\end{Exercise}

\begin{Exercise}
Regn ut med kalkulator:
\begin{center}
\begin{tabular}{l l l l l}
& \textbf{a)} $3\cdot 3\cdot 3\cdot 3\cdot 3\cdot 3$
& \textbf{b)} $5^{8}$
& \textbf{c)} $(-5)^{8}$
\end{tabular}
\end{center}
\end{Exercise}

\begin{Exercise}
Regn ut med kalkulator:
\begin{center}
\begin{tabular}{l l l l l}
& \textbf{a)} $(-4)^5$
& \textbf{b)} $(-4)^6$
& \textbf{c)} $(-3)^5$
& \textbf{d)} $(-3)^6$
\end{tabular}
\end{center}
Klarer du å se på forhånd om svaret blir positivt eller negativt? Forklar!
\end{Exercise}

\end{document}

