\documentclass[a4, 11pt, twoside]{article}

\usepackage{pstricks,pst-plot}
\usepackage[utf8]{inputenc}
\usepackage[english,norsk]{babel}

\usepackage{ulem}

\usepackage{listings}
\usepackage{color}
\usepackage{xcolor}

\usepackage{natbib}


\lstset{ %
  backgroundcolor=\color{white},   % choose the background color; you must add \usepackage{color} or \usepackage{xcolor}
  basicstyle=\footnotesize,        % the size of the fonts that are used for the code
  breakatwhitespace=false,         % sets if automatic breaks should only happen at whitespace
  breaklines=true,                 % sets automatic line breaking
  %captionpos=b,                   % sets the caption-position to bottom
  commentstyle=\color{blue},       % comment style
  deletekeywords={...},            % if you want to delete keywords from the given language
  escapeinside={\%*}{*)},          % if you want to add LaTeX within your code
  extendedchars=true,              % lets you use non-ASCII characters; for 8-bits encodings only, does not work with UTF-8
  keepspaces=true,                 % keeps spaces in text, useful for keeping indentation of code (possibly needs columns=flexible)
  keywordstyle=\color{orange},       % keyword style
  language=Python,                 % the language of the code
  morekeywords={*,...},            % if you want to add more keywords to the set
  rulecolor=\color{black},         % if not set, the frame-color may be changed on line-breaks within not-black text (e.g. comments (green here))
  showspaces=false,                % show spaces everywhere adding particular underscores; it overrides 'showstringspaces'
  showstringspaces=false,          % underline spaces within strings only
  showtabs=false,                  % show tabs within strings adding particular underscores
  stepnumber=2,                    % the step between two line-numbers. If it's 1, each line will be numbered
  stringstyle=\color{violet},      % string literal style
  tabsize=2,                       % sets default tabsize to 2 spaces
}

\usepackage{caption}
\DeclareCaptionFont{white}{\color{white}}
\DeclareCaptionFormat{listing}{\colorbox{gray}{\parbox{\textwidth}{#1#2#3}}}
\captionsetup[lstlisting]{format=listing,labelfont=white,textfont=white}

% This concludes the preamble

\usepackage[colorlinks=true,citecolor=red]{hyperref}

% \usepackage[norsk]{babel}
\usepackage{amsmath, amssymb, amsthm}
\usepackage{makeidx}
\usepackage{graphicx}

\usepackage{fancyhdr}
\pagestyle{fancy}

\usepackage{accents}
\newcommand{\interior}[1]{\accentset{\smash{\raisebox{-0.12ex}{$\scriptstyle\circ$}}}{#1}\rule{0pt}{2.3ex}}
\fboxrule0.0001pt \fboxsep0pt

% used for diagrams, not needed here
% \usepackage{tikz}
% \usepackage{tikz-cd}
% \usetikzlibrary{matrix, arrows, decorations}



%%% New environments
\newtheorem{theorem}{Theorem}[section]
\newtheorem{lemma}[theorem]{Lemma}
\newtheorem{corollary}[theorem]{Corollary}
\newtheorem{prop}[theorem]{Proposition}

\theoremstyle{definition}
\newtheorem{defn}[theorem]{Definition}
\newtheorem{example}[theorem]{Example}
\newtheorem{eksempel}[theorem]{Eksempel}
\newtheorem{exercise}[theorem]{Exercise}
\newtheorem{remark}[theorem]{Remark}
\newtheorem{nremark}[theorem]{Bemerkning}
\newtheorem{advarsel}[theorem]{ADVARSEL}
\newtheorem{question}[theorem]{Question}
\newtheorem{conjecture}[theorem]{Conjecture}
\newtheorem{improvement}[theorem]{Improvement}
\newtheorem{discus}[theorem]{Discus}
\newtheorem{ptheorem}[theorem]{Possible Theorem}
\newtheorem{project}[theorem]{Project}
\newtheorem{solution}[theorem]{Solution}
\newcommand\hra{\hookrightarrow}

%%% Custom definitions and macros


\DeclareMathOperator{\vspan}{Span}
\renewcommand{\d}{\mathrm{\; d}} % for differensialet i integraler.

\newcommand{\Aut}{\mathop{{\rm Aut}}}
\newcommand{\End}{\mathop{{\rm End}}}
\newcommand{\Hom}{\mathop{{\rm Hom}}}
\newcommand{\rank}{\mathop{{\rm rank}}}
\newcommand{\st}{\text{ s.t }}
\renewcommand{\div}{\mathop{{\rm div}}}

\DeclareMathOperator{\Dx}{\frac{\d}{\d x}}
\DeclareMathOperator{\DDx}{\frac{\d^2}{\d x^2}}
\DeclareMathOperator{\erf}{erf}
\DeclareMathOperator{\erfc}{erfc}
\DeclareMathOperator{\sign}{sgn}


% Caligraphic letters
\newcommand{\cA}{\mathcal{A}}
\newcommand{\cB}{\mathcal{B}}
\newcommand{\cR}{\mathcal{R}}
\newcommand{\cC}{\mathcal{C}}
\newcommand{\cD}{\mathcal{D}}
\newcommand{\cE}{\mathcal{E}}
\newcommand{\cF}{\mathcal{F}}
\newcommand{\cG}{\mathcal{G}}
\newcommand{\cH}{\mathcal{H}}
\newcommand{\cI}{\mathcal{I}}
\newcommand{\cJ}{\mathcal{J}}
\newcommand{\cK}{\mathcal{K}}
\newcommand{\cL}{\mathcal{L}}
\newcommand{\cN}{\mathcal{N}}
\newcommand{\cO}{\mathcal{O}}
\newcommand{\cS}{\mathcal{S}}
\newcommand{\cZ}{\mathcal{Z}}
\newcommand{\cP}{\mathcal{P}}
\newcommand{\cT}{\mathcal{T}}
\newcommand{\cU}{\mathcal{U}}
\newcommand{\cV}{\mathcal{V}}
\newcommand{\cX}{\mathcal{X}}
\newcommand{\cW}{\mathcal{W}}

\newcommand{\A}{\mathbb{A}}
\newcommand{\B}{\mathbb{B}}
\newcommand{\C}{\mathbb{C}}
\newcommand{\D}{\mathbb{D}}
\renewcommand{\H}{\mathbb{H}}
\newcommand{\N}{\mathbb{N}}
\newcommand{\K}{\mathbb{K}}
\newcommand{\Q}{\mathbb{Q}}
\newcommand{\Z}{\mathbb{Z}}
\renewcommand{\P}{\mathbb{P}}
\newcommand{\R}{\mathbb{R}}
\newcommand{\U}{\mathbb{U}}
\newcommand{\bT}{\mathbb{T}}
\newcommand{\bD}{\mathbb{D}}

\newcommand{\vu}{\boldsymbol{u}}
\newcommand{\vv}{\boldsymbol{v}}
\newcommand{\vn}{\boldsymbol{n}}
\newcommand{\vf}{\boldsymbol{f}}

\def\di{\partial}
\def\bs{\backslash}
\def\e{\epsilon}
\def\la{\langle}
\def\ra{\rangle}



\renewcommand{\d}{\;\text{d}}
\newcommand{\ocM}{\overline{\mathcal{M}}^{+}}
\newcommand{\ocB}{\overline{\mathcal{B}}}
\newcommand{\ocA}{\overline{\mathcal{A}}}
\newcommand{\Lp}{\text{L}}
\newcommand{\aev}{\text{ a.e }}
\newcommand{\aeev}{\text{-a.e }}

\newcommand{\llra}{\xrightarrow{\hspace*{1cm}}}

\DeclareMathOperator*{\supp}{supp}
\DeclareMathOperator*{\essup}{essup}


\fancyhead[LE, RO]{Matematikk 1P uke 36}
\fancyhead[RE, LO]{Veien om 1}
\setlength{\headheight}{14pt}

\usepackage{exercise}
\def\ExerciseName{Oppgave}

\begin{document}

NB! Oppgaver merket med ''$V$'' inneholder mer veiledning i form
av deloppgaver
enn dere kan forvente på prøver eller eksamen.
Sørg derfor for å klare å løse oppgaver uten slik veiledning. Om du
er usikker på hvordan du kan gå fram, kan du se om du finner en lignende
oppgave merket med ''V'' og gjøre den først.




\section*{Veien om 1}
\subsection*{Repetisjon/oppvarming: brøkregning}
Husk at vi kan forkorte og utvide brøker. Vi kan
gange teller og nevner med samme tall, uten å forandre verdien av brøken.
\begin{eksempel}
\[ \frac{5}{2} = \frac{5\cdot3}{2\cdot3} = \frac{15}{6}\]
Dette vanligvis å utvide brøken.
På akkurat samme måte kan vi forkorte brøker. I eksempelet over har vi da
at
\[\frac{15}{6} = \frac{5\cdot3}{2\cdot3} = \frac{5}{2}\]
Dette blir ofte kalt å forkorte brøken, eller å stryke felles faktorer.
\end{eksempel}
\begin{eksempel}
  For oppgavene med prosent: Husk at prosent betyr hundredel, altså kan
  dere tenke at $\% = \frac{1}{100} = 0,01$. Videre får vi da også at
  $100\% = 1$. Vi tenker vanligvis at $100\%$ svarer til en hel.
  Det er også viktig at dere tenker nøye på hvilken størrelse vi
  skal regne prosenter av. Når for eksempel noe er på tilbud, er det
  \textit{startverdien} vi skal regne prosenter av, altså ordinær pris.
  Dette er viktig siden f.eks 30\% av 100kr er mer enn 30\% av 70kr.
\end{eksempel}

Her er noen videoer dere kan se før dere begynner på oppgavesettet og
underveis:
Introduksjonsvideoer:
\begin{description}
  \item \url{https://www.youtube.com/watch?v=eBKho1HIWT4}
  \item \url{https://www.youtube.com/watch?v=vXlNh9UHmXc}
  \item[finne gammel verdi/prosentgrunnlag:] \url{https://www.youtube.com/watch?v=OILY0sPCHXs}
  \item[finne gammel verdi/prosentgrunnlag:] \url{https://www.youtube.com/watch?v=7-eTUA8Zai8}
  \item[finne gammel verdi/prosentgrunnlag:] \url{https://www.youtube.com/watch?v=WrTCbWKagiA}
\end{description}

\begin{Exercise}
\textit{V}\newline
Per liker å trene. Han betaler 450kr i måneden for et treningsstudio.
I en vanlig måned trener han 25 ganger. Hvor mye betaler han per
trening i en vanlig måned?
\end{Exercise}

\begin{Exercise}
\textit{V}\newline
  Toget mellom Bergen og Oslo kjører i følge \url{http://no.avstand.org/Bergen/Oslo}
  482km på 6 timer og 40min. Hva er gjennomsnittshastigheten til toget?
  \newline \it{HINT: }40min $ = $ 0.666... timer.
\end{Exercise}

\begin{Exercise}
Jens gikk en tur på 2,5 km på 20 min.
\begin{itemize}
\item[\bf a)] Hvor mange minutter brukte Jens på én km?
\item[\bf b)] Hvis han holder samme hastighet,
hvor lang tid bruker han på å gå 3,4km ?
\end{itemize}
\end{Exercise}

\begin{Exercise}
En iskrem kommer i to forskjellige innpakninger.
En koster 24,90kr for 240g, mens en annen koster 
27,70 for 330g. Hvilken pakke lønner det seg å kjøpe?
\end{Exercise}

\begin{Exercise}
En pakke med torsk på 675g fra produsenten ''Fuskeoppdrett AS''
koster 114kr, mens en pakke på 520g fra produsenten ''Sjarkfisk''
koster 90kr. Hvilken produsent har billigst fisk?
\end{Exercise}

\begin{Exercise}
  I en kakeoppskrift trengs det 750g hvetemel til 6 egg.
  Du har bare 2 egg på en søndag, men veldig lyst på kake.
  Hvor mange gram hvetemel trengs det til en oppskrift på 2 egg?
\end{Exercise}

\begin{Exercise}
  I en annen kakeoppskrift trengs det 1400g hvetemel til 7 egg.
  Du har bare 3 egg på en søndag, men veldig lyst på kaken.
  Hvor mange gram hvetemel trengs det til en oppskrift på 3 egg?
\end{Exercise}

\begin{Exercise}
  I en brødoppskrift skal du bruke 850g sammalt hvete og
  400g fint hvetemel. Desverre har du kun 250g fint hvetemel.
  Hvor mye sammalt hvete skal du bruke?
  \newline
  \it{HINT: Regn ut hvor mye sammalt hvetemel du bruker per gram fint hvetemel.}
\end{Exercise}

\begin{Exercise}
Din lokale klesforretning har en bukse på tilbud. Ordinær pris er
420kr, og buksen er satt ned 25\%.
\begin{itemize}
\item[\bf a)] Hvor mye er én prosent av ordinær pris?
\item[\bf b)] Hvor mange kroner utgjorde rabatten?
\item[\bf c)] Hva var tilbudsprisen?
\end{itemize}
\end{Exercise}


\begin{Exercise}
\textit{V}\newline
Din lokale klesforretning har en skjorte på tilbud. Ordinær pris er
540kr, og skjorta er satt ned 20\%.
\begin{itemize}
\item[\bf a)] Hvor mye er én prosent av ordinær pris?
\item[\bf b)] Hvor mange kroner utgjorde rabatten?
\item[\bf c)] Hva var tilbudsprisen?
\end{itemize}
\end{Exercise}

\begin{Exercise}
\textit{V}\newline
  En sportsforretning har 30\% tilbud på fiskestenger. De koster nå 1330 kr.
  \begin{description}
    \item[a)] Forklar at tilbudsprisen utgjør 70\% av ordinær pris
    \item[b)] Forklar at regnestykket \[ \frac{1330\text{kr}}{70} = 19\text{kr} \] forteller oss
      hva 1\% av ordinær pris er.
    \item[c)] Forklar at regnestykket \[\frac{1330\text{kr}}{70}\cdot100 = 19\text{kr}\cdot100 = 1900\text{kr} \]
      forteller oss hva 100\% av ordinær pris er
    \item[d)] Forklar at 100\% av ordinær pris er ordinær pris
  \end{description}
\end{Exercise}


\begin{Exercise}
\textit{V}\newline
Følg stegene i oppgaven over: \newline
Din lokale klesforretning har en skjorte på tilbud. Tilbudspris er
430kr, og skjorta er satt ned 20\% (av ordinær pris).
\begin{itemize}
\item[\bf a)] Forklar at tilbudsprisen utgjør 80\% av ordinær pris
\item[\bf b)] Hvor mye er én prosent av ordinær pris? (gå veien om 1)
\item[\bf c)] Hva var ordinær pris?
\item[\bf d)] Hvor mange kroner utgjorde rabatten?
\end{itemize}
\end{Exercise}

\begin{Exercise}
\textit{V}\newline
  En butikk har satt ned prisen på smågodt med 27\%. Nåprisen er 8kr/hg.
\begin{description}
\item[a)] Forklar at tilbudsprisen utgjør 73\% av ordinær pris
\item[b)] Hvor mye er én prosent av ordinær pris? (gå veien om 1)
\item[c)] Hva var ordinær pris?
\item[d)] Hvor mange kroner utgjorde rabatten?
\end{description}
\end{Exercise}

\begin{Exercise}
\textit{V}\newline
  Prisen på en vare har gått opp 2,5\% iløpet av et år. Nåprisen er 205kr.
\begin{description}
\item[a)] Forklar at nåprisen utgjør 102,5\% av prisen for ett år siden
\item[b)] Hvor mye er én prosent av prisen for ett år siden? (gå veien om 1)
\item[c)] Hva var prisen for ett år siden?
\item[d)] Hvor stor er prisendringen i kroner?
\end{description}
\end{Exercise}

\begin{Exercise}
  Prisen på en vare har gått opp 5\% iløpet av et år. Nåprisen er 1750kr.
\begin{description}
\item[a)] Hva var prisen for ett år siden?
\item[b)] Hvor stor er prisendringen i kroner?
\end{description}
\end{Exercise}

\begin{Exercise}
  En butikk har 25\% tilbud på en vare. Den selges nå til 500kr.
\begin{description}
\item[a)] Hva var ordinær pris?
\item[b)] Hvor stor er rabatten i kroner?
\end{description}
\end{Exercise}

\begin{Exercise}
  Et selskap investerte 30\% av kontantbeholdningen sin i en maskin
  som kostet 150000kr.
  \begin{description}
    \item[a)] Hvor stor var kontantbeholdningen før investeringen?
    \item[b)] Hvor stor er kontantbeholdningen nå?
  \end{description}
\end{Exercise}


\begin{Exercise}
Per skal gå en tur på fjellet. Han har trent i lang tid, men på noe
kortere turer. Han går 20km på 4 timer under treningsøktene.
\begin{itemize}
\item[\bf a)] Hvor lang tid bruker han på 35km med denne farta?
\item[\bf b)] Per regner med å redusere farten med 20\% når han er ute på
    tur. Han skal gå 32km per dag. Hvor lenge må han gå hver dag med denne
        farten ?
\end{itemize}
\end{Exercise}


\begin{Exercise}
Petter skal male en vegg. Han bruker 5L maling på 7m$^2$. Hvor mye
maling bruker han på en vegg med areal 50m$^2$.
\end{Exercise}

\begin{Exercise}
Gjør oppgavene 154, 156, og 157 side 238 i læreboken ''Matematikk 1P, Aschehoug ,2009''.
\end{Exercise}


\section*{Hjernetrim og repetisjon (frivillig)}
\begin{Exercise}
Fyll ut gangetabellen:
\begin{align*}
\begin{tabular}{| l | l | l | l | l | l | l | l | l | l | l | l | l | l |}
\hline
\ \cdot  & \;1\; & \;2\; & \;3\; & \;4\; & \;5\; & \;6\; & \;7\; & \;8\; & \;9\; & \;\;10 & \;\;11 & \;\; 12 & \;\; 13 \\   \hline
\ 1      &   &   &   &   &   &                &   &      &     &   & &  &    \\   \hline
\ 2      &   &   &   &   &   &                &   &      &     &   & &  &    \\   \hline
\ 3      &   &   &   &   &   &                &   &      &     &   & &  &    \\   \hline
\ 4      &   &   &   &   &   &                &   &      &     &   & &  &    \\   \hline
\ 5      &   &   &   &   &   &                &   &      &     &   & &  &    \\   \hline
\ 6      &   &   &   &   &   &                &   &      &     &   & &  &    \\   \hline
\ 7      &   &   &   &   &   &                &   &      &     &   & &  &    \\   \hline
\ 8      &   &   &   &   &   &                &   &      &     &   & &  &    \\   \hline
\ 9      &   &   &   &   &   &                &   &      &     &   & &  &    \\   \hline
\ 10     &   &   &   &   &   &                &   &      &     &   & &  &    \\   \hline
\ 11     &   &   &   &   &   &                &   &      &     &   & &  &    \\   \hline
\ 12     &   &   &   &   &   &                &   &      &     &   & &  &    \\   \hline
\ 13     &   &   &   &   &   &                &   &      &     &   & &  &    \\   \hline
\end{tabular}
\end{align*}
\end{Exercise}

\begin{Exercise}
Regn ut:
\begin{description}
\item[a)] \[\frac{5}{2} -2 + \frac{3}{7}\]
\item[b)] \[\frac{5}{6} -5 + \frac{3}{4}\cdot4\]
\item[c)] \[\frac{5}{3} +3 - \frac{3}{7}\cdot\frac{5}{4}\]
\end{description}
\end{Exercise}

\begin{Exercise}
Vis ved utregning at: \newline
\begin{tabular}{l l l l}
\textbf{a)} $(-3)^3 = -27$ &
\textbf{b)} $(-3)^4 = 81$ &
\textbf{c)} $-3^4 = -81$ &
\textbf{d)} $(-2)^3 = -8$ \\
\textbf{e)} $(-2)^4 = 16$ &
\textbf{f)} $-2^4 = -16$ &
\textbf{g)} $(-10)^2 = 100$ &
\textbf{h)} $-10^2 = -100$
\end{tabular}
\end{Exercise}

\begin{Exercise}
Regn ut:
\begin{description}
\item[a)] $(7 - 8)^{10}$
\item[b)] $(3 - 8)^3 + 5\cdot\sqrt{25}$
\item[c)] $(2 - 4)^4 - 8:4 + 10$
\item[d)] $(13 - 17)^3 + 20:5 + 7\cdot\sqrt{25}$
\end{description}
\end{Exercise}

\begin{Exercise}
Regn ut i hodet, og sjekk med digitalt verktøy: \newline
\begin{tabular}{l l l l}
\textbf{a)} $14\cdot 14$ &
\textbf{b)} $15\cdot 15$ &
\textbf{c)} $16\cdot 16$ &
\textbf{d)} $17\cdot 17$ \\
\textbf{e)} $18\cdot 18$ &
\textbf{f)} $19\cdot 19$ &
\textbf{g)} $21\cdot 21$ &
\textbf{h)} $22\cdot 22$ &
\end{tabular}
\end{Exercise}






\end{document}
