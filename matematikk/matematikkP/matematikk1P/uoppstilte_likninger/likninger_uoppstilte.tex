\documentclass[a4, 11pt, twoside]{article}

\usepackage{pstricks,pst-plot}
\usepackage[utf8]{inputenc}
\usepackage[english,norsk]{babel}

\usepackage{listings}
\usepackage{color}
\usepackage{xcolor}

\usepackage{natbib}


\lstset{ %
  backgroundcolor=\color{white},   % choose the background color; you must add \usepackage{color} or \usepackage{xcolor}
  basicstyle=\footnotesize,        % the size of the fonts that are used for the code
  breakatwhitespace=false,         % sets if automatic breaks should only happen at whitespace
  breaklines=true,                 % sets automatic line breaking
  %captionpos=b,                   % sets the caption-position to bottom
  commentstyle=\color{blue},       % comment style
  deletekeywords={...},            % if you want to delete keywords from the given language
  escapeinside={\%*}{*)},          % if you want to add LaTeX within your code
  extendedchars=true,              % lets you use non-ASCII characters; for 8-bits encodings only, does not work with UTF-8
  keepspaces=true,                 % keeps spaces in text, useful for keeping indentation of code (possibly needs columns=flexible)
  keywordstyle=\color{orange},       % keyword style
  language=Python,                 % the language of the code
  morekeywords={*,...},            % if you want to add more keywords to the set
  rulecolor=\color{black},         % if not set, the frame-color may be changed on line-breaks within not-black text (e.g. comments (green here))
  showspaces=false,                % show spaces everywhere adding particular underscores; it overrides 'showstringspaces'
  showstringspaces=false,          % underline spaces within strings only
  showtabs=false,                  % show tabs within strings adding particular underscores
  stepnumber=2,                    % the step between two line-numbers. If it's 1, each line will be numbered
  stringstyle=\color{violet},      % string literal style
  tabsize=2,                       % sets default tabsize to 2 spaces
}

\usepackage{caption}
\DeclareCaptionFont{white}{\color{white}}
\DeclareCaptionFormat{listing}{\colorbox{gray}{\parbox{\textwidth}{#1#2#3}}}
\captionsetup[lstlisting]{format=listing,labelfont=white,textfont=white}

% This concludes the preamble

\usepackage[colorlinks=true,citecolor=red]{hyperref}

% \usepackage[norsk]{babel}
\usepackage{amsmath, amssymb, amsthm}
\usepackage{makeidx}
\usepackage{graphicx}

\usepackage{fancyhdr}
\pagestyle{fancy}

\usepackage{accents}
\newcommand{\interior}[1]{\accentset{\smash{\raisebox{-0.12ex}{$\scriptstyle\circ$}}}{#1}\rule{0pt}{2.3ex}}
\fboxrule0.0001pt \fboxsep0pt

% used for diagrams, not needed here
% \usepackage{tikz}
% \usepackage{tikz-cd}
% \usetikzlibrary{matrix, arrows, decorations}



%%% New environments
\newtheorem{theorem}{Theorem}[section]
\newtheorem{lemma}[theorem]{Lemma}
\newtheorem{corollary}[theorem]{Corollary}
\newtheorem{prop}[theorem]{Proposition}

\theoremstyle{definition}
\newtheorem{defn}[theorem]{Definition}
\newtheorem{example}[theorem]{Example}
\newtheorem{eksempel}[theorem]{Eksempel}
\newtheorem{exercise}[theorem]{Exercise}
\newtheorem{remark}[theorem]{Remark}
\newtheorem{nremark}[theorem]{Bemerkning}
\newtheorem{question}[theorem]{Question}
\newtheorem{conjecture}[theorem]{Conjecture}
\newtheorem{improvement}[theorem]{Improvement}
\newtheorem{discus}[theorem]{Discus}
\newtheorem{ptheorem}[theorem]{Possible Theorem}
\newtheorem{project}[theorem]{Project}
\newtheorem{solution}[theorem]{Solution}
\newcommand\hra{\hookrightarrow}

%%% Custom definitions and macros


\DeclareMathOperator{\vspan}{Span}
\renewcommand{\d}{\mathrm{\; d}} % for differensialet i integraler.

\newcommand{\Aut}{\mathop{{\rm Aut}}}
\newcommand{\End}{\mathop{{\rm End}}}
\newcommand{\Hom}{\mathop{{\rm Hom}}}
\newcommand{\rank}{\mathop{{\rm rank}}}
\newcommand{\st}{\text{ s.t }}
\renewcommand{\div}{\mathop{{\rm div}}}

\DeclareMathOperator{\Dx}{\frac{\d}{\d x}}
\DeclareMathOperator{\DDx}{\frac{\d^2}{\d x^2}}
\DeclareMathOperator{\erf}{erf}
\DeclareMathOperator{\erfc}{erfc}
\DeclareMathOperator{\sign}{sgn}


% Caligraphic letters
\newcommand{\cA}{\mathcal{A}}
\newcommand{\cB}{\mathcal{B}}
\newcommand{\cR}{\mathcal{R}}
\newcommand{\cC}{\mathcal{C}}
\newcommand{\cD}{\mathcal{D}}
\newcommand{\cE}{\mathcal{E}}
\newcommand{\cF}{\mathcal{F}}
\newcommand{\cG}{\mathcal{G}}
\newcommand{\cH}{\mathcal{H}}
\newcommand{\cI}{\mathcal{I}}
\newcommand{\cJ}{\mathcal{J}}
\newcommand{\cK}{\mathcal{K}}
\newcommand{\cL}{\mathcal{L}}
\newcommand{\cN}{\mathcal{N}}
\newcommand{\cO}{\mathcal{O}}
\newcommand{\cS}{\mathcal{S}}
\newcommand{\cZ}{\mathcal{Z}}
\newcommand{\cP}{\mathcal{P}}
\newcommand{\cT}{\mathcal{T}}
\newcommand{\cU}{\mathcal{U}}
\newcommand{\cV}{\mathcal{V}}
\newcommand{\cX}{\mathcal{X}}
\newcommand{\cW}{\mathcal{W}}

\newcommand{\A}{\mathbb{A}}
\newcommand{\B}{\mathbb{B}}
\newcommand{\C}{\mathbb{C}}
\newcommand{\D}{\mathbb{D}}
\renewcommand{\H}{\mathbb{H}}
\newcommand{\N}{\mathbb{N}}
\newcommand{\K}{\mathbb{K}}
\newcommand{\Q}{\mathbb{Q}}
\newcommand{\Z}{\mathbb{Z}}
\renewcommand{\P}{\mathbb{P}}
\newcommand{\R}{\mathbb{R}}
\newcommand{\U}{\mathbb{U}}
\newcommand{\bT}{\mathbb{T}}
\newcommand{\bD}{\mathbb{D}}

\newcommand{\vu}{\boldsymbol{u}}
\newcommand{\vv}{\boldsymbol{v}}
\newcommand{\vn}{\boldsymbol{n}}
\newcommand{\vf}{\boldsymbol{f}}

\def\di{\partial}
\def\bs{\backslash}
\def\e{\epsilon}
\def\la{\langle}
\def\ra{\rangle}



\renewcommand{\d}{\;\text{d}}
\newcommand{\ocM}{\overline{\mathcal{M}}^{+}}
\newcommand{\ocB}{\overline{\mathcal{B}}}
\newcommand{\ocA}{\overline{\mathcal{A}}}
\newcommand{\Lp}{\text{L}}
\newcommand{\aev}{\text{ a.e }}
\newcommand{\aeev}{\text{-a.e }}

\newcommand{\llra}{\xrightarrow{\hspace*{1cm}}}

\DeclareMathOperator*{\supp}{supp}
\DeclareMathOperator*{\essup}{essup}


\fancyhead[LE, RO]{Algebra}
\fancyhead[RE, LO]{Uoppstilte likninger}
\setlength{\headheight}{14pt}

\usepackage{exercise}
\def\ExerciseName{Oppgave}


\begin{document}
\begin{Exercise}
Løs likningene med hensyn på $x$ ved regning: \newline
\begin{itemize}
\item[\bf a)] \[ \frac{x}{3} = \frac{2}{3} \]
\item[\bf b)] \[ \frac{x}{5} = \frac{2}{3} \]
\item[\bf c)] \[8(8x - 6) + 14 = 94 \]
\item[\bf d)] \[-11(7x - 2) + 18 = -37\]
\end{itemize}
\end{Exercise}


\section*{Uoppstilte likninger}
Det er viktig at dere leser teksten nøye. Bruk gjerne
følgende tabell til å tolke teksten

\begin{center}
\begin{tabular}{|l | l | l |}
\hline
fem mindre enn $x$ & $x - 5$ \\ \hline 
fem mer enn $x$ &  $x + 5$ \\ \hline
fem ganger så mye som $x$ & $5x$ \\ \hline
halvparten av $x$ & $\frac{x}{2}$ \\ \hline
$x$ sjudeler & $\frac{x}{7}$ \\ \hline
\end{tabular}
\end{center}
\begin{eksempel}
  Vi ser på en typisk oppgave: \newline
Anders har tre ganger så mange penger som
 Kari. Anta at Kari har $x$ kroner.
\begin{itemize}
 \item[\bf a)] Sett opp et utrykk som beskriver hvor mye penger Kari og Anders har til sammen
 \item[\bf b)] Anta at de til sammen har 1000 kr. Hvor mye penger har hver av dem?
\end{itemize}
LØSNING: Siden oppgaven sier at Kari har $x$ kroner, og at Anders har
tre ganger så mange kroner, må altså Anders ha $3\cdot x = 3x$ kroner.
Utrykket som forteller hvor mye de har til sammen, er selvsagt pengene
til Kari og Anders lagt sammen:
\begin{itemize}
  \item[\bf a)] Kari sine penger + Anders sine penger $ = x + 3x = 4x$.
  \item[\bf b)] Til sammen skal de ha 1000kr. Altså må
    \[4x = 1000 \]
    Vi løser likningen og får da
    \begin{align*}
      4x & = 1000 \\
      \frac{4x}{4} & = \frac{1000}{4} \\
      x & = 250
    \end{align*}
\begin{itemize}
\end{eksempel}

\begin{Exercise}
Anders har dobbelt så mange penger som
 Kari. Anta at Kari har $x$ kroner.
\begin{itemize}
 \item[\bf a)] Sett opp et utrykk som beskriver hvor mye penger Kari og Anders har til sammen
 \item[\bf b)] Anta at de til sammen har 1000 kr. Hvor mye penger har hver av dem?
\end{itemize}
\end{Exercise}

\begin{Exercise}
Ove har 200 kroner mer enn
 Per. Anta at Per har $x$ kroner.
\begin{itemize}
 \item[\bf a)] Sett opp et utrykk som beskriver hvor mye penger Ove og Per har til sammen
 \item[\bf b)] Anta at de til sammen har 1200 kr. Hvor mye penger har hver av dem?
\end{itemize}
\end{Exercise}

\begin{Exercise}
Anders har dobbelt så mange penger som
 Anne. Kari har 400 kr mer enn Anne. Anta at Anne har $x$ kroner.
\begin{itemize}
 \item[\bf a)] Sett opp et utrykk som beskriver hvor mye penger Kari, Anders og Anne har til 
 sammen.
 \item[\bf b)] Anta at de til sammen har 5200 kr. Hvor mye penger har hver av dem?
\end{itemize}
\end{Exercise}

\begin{Exercise}
Anders har dobbelt så mange penger som Anne. 
Kari har 1200 kr mer enn Anne.
\begin{itemize}
 \item[\bf a)] Sett opp et utrykk som beskriver hvor mye penger Kari, Anders og Anne har til 
 sammen.
 \item[\bf b)] Anta at de til sammen har 7600 kr. Hvor mye penger har hver av dem?
\end{itemize}
\end{Exercise}

\begin{Exercise}
  Ole er 3 ganger så gammel som Pål, som er fire år eldre enn Jens.
  Til sammen er Ole, Jens og Pål 146 år gamle. Hvor gamle er hver av dem?
\end{Exercise}

\begin{Exercise}
Marte, Kåre og Johanne arver formue. Marte skal arve dobbelt så mye som
Kåre, som skal arve 60000 kroner mindre enn Johanne. Til sammen skal
de arve 360 000 kroner. Hvor mye arver hver av dem ?
\end{Exercise}

\begin{Exercise}
I fotball får et lag tre poeng for seier og ett poeng for uavgjort.
I fjor vant laget til Odd ni flere kamper enn de spilte uavgjort. De
fikk til sammen 67 poeng. \newline
Hvor mange kamper vant laget til Odd?
\end{Exercise}

\begin{Exercise}
I et erstatningsoppgjør i Andeby skal de kriminelle i B-gjengen 
betale erstatning etter bruddets grovhet. I påvente av konkrete tall
fordeler erstatningsansvaret seg omtrent slik:
\begin{itemize}
\item Kriminell 1 betaler dobbelt så mye som kriminell 2
\item Kriminell 2 betaler 30 000 mindre enn kriminell 3
\end{itemize}
\begin{description}
\item[a)] Anta at det totale er 500 000 kroner. Hvor mye må hver persjon betale ?
\item[b)] Anta at det totale erstatningsansvaret er $K$ kroner. Lag tre formler/funksjoner for hvor
          mye hver persjon må betale.
\item[c)] Plot grafene til funksjonene du fant i oppgave b) i geogebra
\end{description}
\end{Exercise}

\begin{Exercise}
Til slutt et skikkelig udyr: \newline
 En gruppe på fire personer bestående av Åse, Bob, Janne og Fredrik skal
 ut å reise. Til sammen trenger de
 $8,9343\cdot10^4\text{kr}$.
 \begin{itemize}
  \item Åse betaler 3,5 ganger så mye som Kåre
  \item Bob betaler 244 kroner mer enn Åse
  \item Janne betaler halvparten så mye som Kåre
  \item Fredrik betaler $\frac{7}{2}$ ganger så mye som Janne.
 \end{itemize}
 Sett opp en likning som beskriver situasjonen. Forklar hvilken
 person hvert ledd representerer, og finn ut hvor mye penger hver
 av dem betaler.
\end{itemize}
\end{Exercise}

\end{document}
