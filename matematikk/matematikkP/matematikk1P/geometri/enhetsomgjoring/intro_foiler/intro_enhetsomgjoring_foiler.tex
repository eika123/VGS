\documentclass{beamer}

\usepackage[utf8]{inputenc}
\usepackage[norsk]{babel}

\usepackage{qtree}

\usepackage{tikz}
\usetikzlibrary{fit, arrows.meta}

\definecolor{myorange}{RGB}{206,103,44}
\definecolor{myred}{RGB}{176,24,43}
\definecolor{mybg}{RGB}{166,161,121}

\pgfdeclarelayer{background}
\pgfdeclarelayer{backgroundii}
\pgfsetlayers{backgroundii,background,main}

\usepackage{amsmath, amssymb}
\usepackage[misc]{ifsym}
%%% New environments

%\newtheorem{theorem}{Theorem}[section]
%\newtheorem{lemma}[theorem]{Lemma}
%\newtheorem{corollary}[theorem]{Corollary}
%\newtheorem{prop}[theorem]{Proposition}

\newtheorem{teorem}{Teorem}[section]
\newtheorem{nlemma}[theorem]{Lemma}
\newtheorem{korollar}[theorem]{Korollar}
\newtheorem{setn}[theorem]{Setning}

\theoremstyle{definition}
\newtheorem{defn}[theorem]{Definition}
\newtheorem{ndefn}[theorem]{Definisjon}
%\newtheorem{example}[theorem]{Example} \newtheorem{eksempel}[theorem]{Eksempel}
\newtheorem{eksempelx}{Eksempel}
\newenvironment{eksempel}
   {\pushQED{\qed}\renewcommand{\qedsymbol}{%\textbf{eksempel slutt:}
   $\clubsuit$}\eksempelx}
   {\popQED\endeksempelx}
\setcounter{eksempel}{0}


\newtheorem{oppgave}{Oppgave}
\setcounter{oppgave}{0}

\newtheorem{exercise}[theorem]{Exercise}
\newtheorem{remark}[theorem]{Remark}
\newtheorem{nremark}[theorem]{Bemerkning}
\newtheorem{question}[theorem]{Question}
\newtheorem{conjecture}[theorem]{Conjecture}
\newtheorem{improvement}[theorem]{Improvement}
\newtheorem{discus}[theorem]{Discus}
\newtheorem{ptheorem}[theorem]{Possible Theorem}
\newtheorem{project}[theorem]{Project}
%\newtheorem{solution}[theorem]{Solution}


%\usetheme{Warsaw}
%\usetheme{Antibes}
%\usetheme{Bergen}
%\usetheme{Berkeley}
%\usetheme{Berlin}
%\usetheme{Copenhagen}
%\usetheme{Darmstadt}
%\usetheme{Dresden}
%\usetheme{Frankfurt}
%\usetheme{Goettingen}
%\usetheme{Hannover}
%\usetheme{Ilmenau}
%\usetheme{JuanLesPins}
%\usetheme{Luebeck}
%\usetheme{Madrid}
%\usetheme{Malmoe}
%\usetheme{Marburg}
%\usetheme{Montpellier}
\usetheme{PaloAlto}
%\usetheme{Pittsburgh}
%\usetheme{Rochester}
%\usetheme{Singapore}
%\usetheme{Szeged}
%\usetheme{boxes}
%\usetheme{default}
%\usetheme{CambridgeUS}

% Caligraphic letters
\newcommand{\cA}{\mathcal{A}}
\newcommand{\cB}{\mathcal{B}}
\newcommand{\cR}{\mathcal{R}}
\newcommand{\cC}{\mathcal{C}}
\newcommand{\cD}{\mathcal{D}}
\newcommand{\cE}{\mathcal{E}}
\newcommand{\cF}{\mathcal{F}}
\newcommand{\cG}{\mathcal{G}}
\newcommand{\cH}{\mathcal{H}}
\newcommand{\cI}{\mathcal{I}}
\newcommand{\cJ}{\mathcal{J}}
\newcommand{\cK}{\mathcal{K}}
\newcommand{\cL}{\mathcal{L}}
\newcommand{\cN}{\mathcal{N}}
\newcommand{\cO}{\mathcal{O}}
\newcommand{\cS}{\mathcal{S}}
\newcommand{\cZ}{\mathcal{Z}}
\newcommand{\cP}{\mathcal{P}}
\newcommand{\cT}{\mathcal{T}}
\newcommand{\cU}{\mathcal{U}}
\newcommand{\cV}{\mathcal{V}}
\newcommand{\cX}{\mathcal{X}}
\newcommand{\cW}{\mathcal{W}}


\tikzset{
  hpil/.style={
          ->,
          thick,
          shorten <=2pt,
          shorten =>2pt,}
}
\tikzset{
  vpil/.style={
          <-,
          thick,
          shorten =>2pt,}
}



\title{Lengde-enheter}
\date{\today}
\subject{Mathematics}

\begin{document}

\frame{\titlepage}

\begin{frame}
    \frametitle{Mål}
    Læreplanmål i geometri: Mål for opplæringa er at eleven skal kunne:
    \begin{itemize}
        \item \textit{Rekne med ulike måleeiningar, bruke ulike målereiskaper, vurdere kva for målereiskaper
            som er formålstenlege}, og vurdere kor usikre målingane er.
    \end{itemize}
    Timens læringsmål:
    \begin{itemize}
        \item Kunne gjøre om mellom ulike metriske lengdeenheter
        \item Kunne vurdere hvilken metrisk måleenhet som er passende (formålstenleg).
        \item Kunne gjøre om mellom metriske og ikke-metriske lengdeenheter.
    \end{itemize}
        
\end{frame}


\section{Lengdeenheter: Meteren}

%\begin{frame}
%\frametitle{Hvor finner vi forholdstall}
%Eksempler på forholdstall
%\begin{itemize}
%\item<2-> Kilopris, literpris, (kroner \textbf{per} kilogram, kroner \textbf{per} liter)
%\item<3-> Blande saft (råsaft \textbf{per} liter, en \textit{ konsentrasjon })
%\item<4-> Trykk (kraft \textbf{per} kvadratmeter)
%\item<5-> kronekurs (veksle penger: euro \textbf{per} krone, kroner \textbf{per} euro)
%\item<6-> Trafikk\textbf{tetthet}: kjøretøy  \textbf{per} time.
%\item<7-> Masse\textbf{tetthet}: kilogram \textbf{per} kubikkmeter
%\item<8-> Baking: dL mel  \textbf{per} dL melk
%\item<9-> Hastighet: meter \textbf{per} sekund
%\item<10-> Kan du finne flere eksempler ?
%\end{itemize}
%\end{frame}



%\begin{frame}
%\frametitle{Meteren: visualisering av oppdeling}
%\begin{center}
%\begin{tikzpicture}[line cap=round,line join=round,>=triangle 45,x=6cm,y=1cm]
%\draw (0, 0) -- (1.0, 0);
%
%\foreach \x in {0, 0.1, 0.2, 0.3, 0.4, 0.5, 0.6, 0.7, 0.8, 0.9, 1},
%\pgfmathsetmacro\result{\x*10}
%    \draw[shift={(\x,0)},color=black] (0pt,2pt) -- (0pt,-2pt) node[below] {\tiny$\pgfmathprintnumber{\result}$dm};
%
%\foreach \x in {0, 0.01, 0.02, 0.03, 0.04, 0.05, 0.06, 0.07, 0.08, 0.09, 0.1,
%                0.11, 0.12, 0.13, 0.14, 0.15, 0.16, 0.17, 0.18, 0.19, 0.2,
%                0.21, 0.22, 0.23, 0.24, 0.25, 0.26, 0.27, 0.28, 0.29, 0.3, 
%                0.31, 0.32, 0.33, 0.34, 0.35, 0.36, 0.37, 0.38, 0.39, 0.3, 
%                0.41, 0.42, 0.43, 0.44, 0.45, 0.46, 0.47, 0.48, 0.49, 0.4, 
%                0.51, 0.52, 0.53, 0.54, 0.55, 0.56, 0.57, 0.58, 0.59, 0.5, 
%                0.61, 0.62, 0.63, 0.64, 0.65, 0.66, 0.67, 0.68, 0.69, 0.6, 
%                0.71, 0.72, 0.73, 0.74, 0.75, 0.76, 0.77, 0.78, 0.79, 0.7, 
%                0.81, 0.82, 0.83, 0.84, 0.85, 0.86, 0.87, 0.88, 0.89, 0.9, 
%                0.91, 0.92, 0.93, 0.94, 0.95, 0.96, 0.97, 0.98, 0.99, 1 }
%\draw[shift={(\x,0)},color=black] (0pt,1pt) -- (0pt,-1pt);
%
%
%\end{tikzpicture}
%\end{center}
%\end{frame}
%
%\begin{frame}
%\frametitle{Meteren: visualisering av oppdeling}
%\begin{center}
%\begin{tikzpicture}[line cap=round,line join=round,>=triangle 45,x=6cm,y=1cm]
%\draw (0, 0) -- (1.0, 0);
%
%
%\foreach \x in {0, 0.1, 0.2, 0.3, 0.4, 0.5, 0.6, 0.7, 0.8, 0.9, 1},
%\pgfmathsetmacro\result{\x*10}
%    \draw[shift={(\x,0)},color=black] (0pt,2pt) -- (0pt,-2pt) node[below] {\tiny$\pgfmathprintnumber{\result}$dm};
%
%\foreach \x in {0, 0.01, 0.02, 0.03, 0.04, 0.05, 0.06, 0.07, 0.08, 0.09, 0.1,
%                0.11, 0.12, 0.13, 0.14, 0.15, 0.16, 0.17, 0.18, 0.19, 0.2,
%                0.21, 0.22, 0.23, 0.24, 0.25, 0.26, 0.27, 0.28, 0.29, 0.3, 
%                0.31, 0.32, 0.33, 0.34, 0.35, 0.36, 0.37, 0.38, 0.39, 0.3, 
%                0.41, 0.42, 0.43, 0.44, 0.45, 0.46, 0.47, 0.48, 0.49, 0.4, 
%                0.51, 0.52, 0.53, 0.54, 0.55, 0.56, 0.57, 0.58, 0.59, 0.5, 
%                0.61, 0.62, 0.63, 0.64, 0.65, 0.66, 0.67, 0.68, 0.69, 0.6, 
%                0.71, 0.72, 0.73, 0.74, 0.75, 0.76, 0.77, 0.78, 0.79, 0.7, 
%                0.81, 0.82, 0.83, 0.84, 0.85, 0.86, 0.87, 0.88, 0.89, 0.9, 
%                0.91, 0.92, 0.93, 0.94, 0.95, 0.96, 0.97, 0.98, 0.99, 1 }
%\draw[shift={(\x,0)},color=black] (0pt,1pt) -- (0pt,-1pt);
%
%    \draw[color=red] (0.05, 0) circle (0.3cm);
%
%\end{tikzpicture}
%\end{center}
%\end{frame}
%
%\begin{frame}
%\frametitle{Meteren: visualisering av oppdeling}
%\begin{center}
%\begin{tikzpicture}[line cap=round,line join=round,>=triangle 45,x=6cm,y=1cm]
%\draw (0, 0) -- (1.0, 0);
%
%\foreach \x in {0, 0.1, 0.2, 0.3, 0.4, 0.5, 0.6, 0.7, 0.8, 0.9, 1},
%\pgfmathsetmacro\result{\x*10}
%    \draw[shift={(\x,0)},color=black] (0pt,2pt) -- (0pt,-2pt) node[below] {\tiny$\pgfmathprintnumber{\result}$dm};
%
%\foreach \x in {0, 0.01, 0.02, 0.03, 0.04, 0.05, 0.06, 0.07, 0.08, 0.09, 0.1,
%                0.11, 0.12, 0.13, 0.14, 0.15, 0.16, 0.17, 0.18, 0.19, 0.2,
%                0.21, 0.22, 0.23, 0.24, 0.25, 0.26, 0.27, 0.28, 0.29, 0.3, 
%                0.31, 0.32, 0.33, 0.34, 0.35, 0.36, 0.37, 0.38, 0.39, 0.3, 
%                0.41, 0.42, 0.43, 0.44, 0.45, 0.46, 0.47, 0.48, 0.49, 0.4, 
%                0.51, 0.52, 0.53, 0.54, 0.55, 0.56, 0.57, 0.58, 0.59, 0.5, 
%                0.61, 0.62, 0.63, 0.64, 0.65, 0.66, 0.67, 0.68, 0.69, 0.6, 
%                0.71, 0.72, 0.73, 0.74, 0.75, 0.76, 0.77, 0.78, 0.79, 0.7, 
%                0.81, 0.82, 0.83, 0.84, 0.85, 0.86, 0.87, 0.88, 0.89, 0.9, 
%                0.91, 0.92, 0.93, 0.94, 0.95, 0.96, 0.97, 0.98, 0.99, 1 }
%\draw[shift={(\x,0)},color=black] (0pt,1pt) -- (0pt,-1pt);
%
%    \draw[color=red] (0.05, 0) circle (0.3cm);
%
%
%\draw (0, -1.5) -- (1.0, -1.5);
%
%\draw[color=red] (0,0) -- (0, -1.5);
%\draw[color=red] (0.1, 0) -- (1, -1.5);
%
%\foreach \x in {0, 0.1, 0.2, 0.3, 0.4, 0.5, 0.6, 0.7, 0.8, 0.9, 1}
%\pgfmathsetmacro\result{\x*10}
%    \draw[shift={(\x,-1.5)},color=black] (0pt,2pt) -- (0pt,-2pt) node[below] {\tiny$\pgfmathprintnumber{\result}$cm};
%
%\end{tikzpicture}
%\end{center}
%\end{frame}
%
%\begin{frame}
%\frametitle{Meteren: visualisering av oppdeling}
%\begin{center}
%\begin{tikzpicture}[line cap=round,line join=round,>=triangle 45,x=6cm,y=1cm]
%\draw (0, 0) -- (1.0, 0);
%
%\foreach \x in {0, 0.1, 0.2, 0.3, 0.4, 0.5, 0.6, 0.7, 0.8, 0.9, 1},
%\pgfmathsetmacro\result{\x*10}
%    \draw[shift={(\x,0)},color=black] (0pt,2pt) -- (0pt,-2pt) node[below] {\tiny$\pgfmathprintnumber{\result}$dm};
%
%\foreach \x in {0, 0.01, 0.02, 0.03, 0.04, 0.05, 0.06, 0.07, 0.08, 0.09, 0.1,
%                0.11, 0.12, 0.13, 0.14, 0.15, 0.16, 0.17, 0.18, 0.19, 0.2,
%                0.21, 0.22, 0.23, 0.24, 0.25, 0.26, 0.27, 0.28, 0.29, 0.3, 
%                0.31, 0.32, 0.33, 0.34, 0.35, 0.36, 0.37, 0.38, 0.39, 0.3, 
%                0.41, 0.42, 0.43, 0.44, 0.45, 0.46, 0.47, 0.48, 0.49, 0.4, 
%                0.51, 0.52, 0.53, 0.54, 0.55, 0.56, 0.57, 0.58, 0.59, 0.5, 
%                0.61, 0.62, 0.63, 0.64, 0.65, 0.66, 0.67, 0.68, 0.69, 0.6, 
%                0.71, 0.72, 0.73, 0.74, 0.75, 0.76, 0.77, 0.78, 0.79, 0.7, 
%                0.81, 0.82, 0.83, 0.84, 0.85, 0.86, 0.87, 0.88, 0.89, 0.9, 
%                0.91, 0.92, 0.93, 0.94, 0.95, 0.96, 0.97, 0.98, 0.99, 1 }
%\draw[shift={(\x,0)},color=black] (0pt,1pt) -- (0pt,-1pt);
%
%    \draw[color=red] (0.05, 0) circle (0.3cm);
%
%
%\draw (0, -1.5) -- (1.0, -1.5);
%
%\draw[color=red] (0,0) -- (0, -1.5);
%\draw[color=red] (0.1, 0) -- (1, -1.5);
%
%\foreach \x in {0, 0.1, 0.2, 0.3, 0.4, 0.5, 0.6, 0.7, 0.8, 0.9, 1}
%\pgfmathsetmacro\result{\x*10}
%    \draw[shift={(\x,-1.5)},color=black] (0pt,2pt) -- (0pt,-2pt) node[below] {\tiny$\pgfmathprintnumber{\result}$cm};
%
%
%\foreach \z in {0, 0.01, 0.02, 0.03, 0.04, 0.05, 0.06, 0.07, 0.08, 0.09, 0.1,
%                   0.11, 0.12, 0.13, 0.14, 0.15, 0.16, 0.17, 0.18, 0.19, 0.2,
%                   0.21, 0.22, 0.23, 0.24, 0.25, 0.26, 0.27, 0.28, 0.29, 0.3, 
%                   0.31, 0.32, 0.33, 0.34, 0.35, 0.36, 0.37, 0.38, 0.39, 0.3, 
%                   0.41, 0.42, 0.43, 0.44, 0.45, 0.46, 0.47, 0.48, 0.49, 0.4, 
%                   0.51, 0.52, 0.53, 0.54, 0.55, 0.56, 0.57, 0.58, 0.59, 0.5, 
%                   0.61, 0.62, 0.63, 0.64, 0.65, 0.66, 0.67, 0.68, 0.69, 0.6, 
%                   0.71, 0.72, 0.73, 0.74, 0.75, 0.76, 0.77, 0.78, 0.79, 0.7, 
%                   0.81, 0.82, 0.83, 0.84, 0.85, 0.86, 0.87, 0.88, 0.89, 0.9, 
%                   0.91, 0.92, 0.93, 0.94, 0.95, 0.96, 0.97, 0.98, 0.99, 1 }
%\draw[shift={(\z, -1.5)},color=black] (0pt,1pt) -- (0pt,-1pt);
%
%\end{tikzpicture}
%\end{center}
%\end{frame}
%
%
%\begin{frame}
%\frametitle{Meteren: visualisering av oppdeling}
%\begin{center}
%\begin{tikzpicture}[line cap=round,line join=round,>=triangle 45,x=6cm,y=1cm]
%\draw (0, 0) -- (1.0, 0);
%
%\foreach \x in {0, 0.1, 0.2, 0.3, 0.4, 0.5, 0.6, 0.7, 0.8, 0.9, 1},
%\pgfmathsetmacro\result{\x*10}
%    \draw[shift={(\x,0)},color=black] (0pt,2pt) -- (0pt,-2pt) node[below] {\tiny$\pgfmathprintnumber{\result}$dm};
%
%\foreach \x in {0, 0.01, 0.02, 0.03, 0.04, 0.05, 0.06, 0.07, 0.08, 0.09, 0.1,
%                0.11, 0.12, 0.13, 0.14, 0.15, 0.16, 0.17, 0.18, 0.19, 0.2,
%                0.21, 0.22, 0.23, 0.24, 0.25, 0.26, 0.27, 0.28, 0.29, 0.3, 
%                0.31, 0.32, 0.33, 0.34, 0.35, 0.36, 0.37, 0.38, 0.39, 0.3, 
%                0.41, 0.42, 0.43, 0.44, 0.45, 0.46, 0.47, 0.48, 0.49, 0.4, 
%                0.51, 0.52, 0.53, 0.54, 0.55, 0.56, 0.57, 0.58, 0.59, 0.5, 
%                0.61, 0.62, 0.63, 0.64, 0.65, 0.66, 0.67, 0.68, 0.69, 0.6, 
%                0.71, 0.72, 0.73, 0.74, 0.75, 0.76, 0.77, 0.78, 0.79, 0.7, 
%                0.81, 0.82, 0.83, 0.84, 0.85, 0.86, 0.87, 0.88, 0.89, 0.9, 
%                0.91, 0.92, 0.93, 0.94, 0.95, 0.96, 0.97, 0.98, 0.99, 1 }
%\draw[shift={(\x,0)},color=black] (0pt,1pt) -- (0pt,-1pt);
%
%    \draw[color=red] (0.05, 0) circle (0.3cm);
%
%
%\draw (0, -1.5) -- (1.0, -1.5);
%
%\draw[color=red] (0,0) -- (0, -1.5);
%\draw[color=red] (0.1, 0) -- (1, -1.5);
%
%\foreach \x in {0, 0.1, 0.2, 0.3, 0.4, 0.5, 0.6, 0.7, 0.8, 0.9, 1}
%\pgfmathsetmacro\result{\x*10}
%    \draw[shift={(\x,-1.5)},color=black] (0pt,2pt) -- (0pt,-2pt) node[below] {\tiny$\pgfmathprintnumber{\result}$cm};
%
%
%\foreach \z in {0, 0.01, 0.02, 0.03, 0.04, 0.05, 0.06, 0.07, 0.08, 0.09, 0.1,
%                   0.11, 0.12, 0.13, 0.14, 0.15, 0.16, 0.17, 0.18, 0.19, 0.2,
%                   0.21, 0.22, 0.23, 0.24, 0.25, 0.26, 0.27, 0.28, 0.29, 0.3, 
%                   0.31, 0.32, 0.33, 0.34, 0.35, 0.36, 0.37, 0.38, 0.39, 0.3, 
%                   0.41, 0.42, 0.43, 0.44, 0.45, 0.46, 0.47, 0.48, 0.49, 0.4, 
%                   0.51, 0.52, 0.53, 0.54, 0.55, 0.56, 0.57, 0.58, 0.59, 0.5, 
%                   0.61, 0.62, 0.63, 0.64, 0.65, 0.66, 0.67, 0.68, 0.69, 0.6, 
%                   0.71, 0.72, 0.73, 0.74, 0.75, 0.76, 0.77, 0.78, 0.79, 0.7, 
%                   0.81, 0.82, 0.83, 0.84, 0.85, 0.86, 0.87, 0.88, 0.89, 0.9, 
%                   0.91, 0.92, 0.93, 0.94, 0.95, 0.96, 0.97, 0.98, 0.99, 1 }
%\draw[shift={(\z, -1.5)},color=black] (0pt,1pt) -- (0pt,-1pt);
%
%
%\draw[color=red] (0.05, -1.5) circle (0.3cm);
%
%\end{tikzpicture}
%\end{center}
%\end{frame}



\begin{frame}
\frametitle{Meteren: visualisering av oppdeling}
\begin{center}
\begin{tikzpicture}[line cap=round,line join=round,>=triangle 45,x=6cm,y=1cm]
\draw (0, 0) -- (1.0, 0);

\foreach \x in {0, 0.1, 0.2, 0.3, 0.4, 0.5, 0.6, 0.7, 0.8, 0.9, 1},
\pgfmathsetmacro\result{\x*10}
    \draw[shift={(\x,0)},color=black] (0pt,2pt) -- (0pt,-2pt) node[below] {\tiny$\pgfmathprintnumber{\result}$dm};

\pause

\foreach \x in {0, 0.01, 0.02, 0.03, 0.04, 0.05, 0.06, 0.07, 0.08, 0.09, 0.1,
                0.11, 0.12, 0.13, 0.14, 0.15, 0.16, 0.17, 0.18, 0.19, 0.2,
                0.21, 0.22, 0.23, 0.24, 0.25, 0.26, 0.27, 0.28, 0.29, 0.3, 
                0.31, 0.32, 0.33, 0.34, 0.35, 0.36, 0.37, 0.38, 0.39, 0.3, 
                0.41, 0.42, 0.43, 0.44, 0.45, 0.46, 0.47, 0.48, 0.49, 0.4, 
                0.51, 0.52, 0.53, 0.54, 0.55, 0.56, 0.57, 0.58, 0.59, 0.5, 
                0.61, 0.62, 0.63, 0.64, 0.65, 0.66, 0.67, 0.68, 0.69, 0.6, 
                0.71, 0.72, 0.73, 0.74, 0.75, 0.76, 0.77, 0.78, 0.79, 0.7, 
                0.81, 0.82, 0.83, 0.84, 0.85, 0.86, 0.87, 0.88, 0.89, 0.9, 
                0.91, 0.92, 0.93, 0.94, 0.95, 0.96, 0.97, 0.98, 0.99, 1 }
\draw[shift={(\x,0)},color=black] (0pt,1pt) -- (0pt,-1pt);

\pause

    \draw[color=red] (0.05, 0) circle (0.3cm);


\pause

\draw (0, -1.5) -- (1.0, -1.5);

\draw[color=red] (0,0) -- (0, -1.5);
\draw[color=red] (0.1, 0) -- (1, -1.5);


\foreach \x in {0, 0.1, 0.2, 0.3, 0.4, 0.5, 0.6, 0.7, 0.8, 0.9, 1}
\pgfmathsetmacro\result{\x*10}
    \draw[shift={(\x,-1.5)},color=black] (0pt,2pt) -- (0pt,-2pt) node[below] {\tiny$\pgfmathprintnumber{\result}$cm};

\foreach \z in {0, 0.01, 0.02, 0.03, 0.04, 0.05, 0.06, 0.07, 0.08, 0.09, 0.1,
                   0.11, 0.12, 0.13, 0.14, 0.15, 0.16, 0.17, 0.18, 0.19, 0.2,
                   0.21, 0.22, 0.23, 0.24, 0.25, 0.26, 0.27, 0.28, 0.29, 0.3, 
                   0.31, 0.32, 0.33, 0.34, 0.35, 0.36, 0.37, 0.38, 0.39, 0.3, 
                   0.41, 0.42, 0.43, 0.44, 0.45, 0.46, 0.47, 0.48, 0.49, 0.4, 
                   0.51, 0.52, 0.53, 0.54, 0.55, 0.56, 0.57, 0.58, 0.59, 0.5, 
                   0.61, 0.62, 0.63, 0.64, 0.65, 0.66, 0.67, 0.68, 0.69, 0.6, 
                   0.71, 0.72, 0.73, 0.74, 0.75, 0.76, 0.77, 0.78, 0.79, 0.7, 
                   0.81, 0.82, 0.83, 0.84, 0.85, 0.86, 0.87, 0.88, 0.89, 0.9, 
                   0.91, 0.92, 0.93, 0.94, 0.95, 0.96, 0.97, 0.98, 0.99, 1 }
\draw[shift={(\z, -1.5)},color=black] (0pt,1pt) -- (0pt,-1pt);

\pause


\draw[color=red] (0.05, -1.5) circle (0.3cm);

\pause


\draw[color=red] (0,-1.5) -- (0, -3.0);
\draw[color=red] (0.1, -1.5) -- (1, -3.0);

\draw (0, -3.0) -- (1, -3.0);


\foreach \x in {0, 0.1, 0.2, 0.3, 0.4, 0.5, 0.6, 0.7, 0.8, 0.9, 1}
\pgfmathsetmacro\result{\x*10}
    \draw[shift={(\x,-3.0)},color=black] (0pt,2pt) -- (0pt,-2pt) node[below] {\tiny$\pgfmathprintnumber{\result}$mm};


\end{tikzpicture}
\end{center}

\visible<7->{Kontrollspørsmål: Hvor mange millimeter er det i 0,5dm ?}
\end{frame}


\begin{frame}
    \frametitle{Prefikser: De mest brukte}
    Dere har kanskje hørt at ''kilo betyr tusen'', eller at ''centi betyr hundredel''. Det er fordi disse
    ordene kalles \textit{prefikser}
    \begin{center}
    \begin{tabular}{l l l}
        \textbf{prefiks} & \textbf{forkortelse / symbol} & \textbf{verdi}            \\ \hline
        Kilo             & k                             & 1000                      \\
        Hekto            & h                             & 100                       \\
        deka             & da                            & 10                        \\
                         &                               & 1                         \\
        desi             & d                             & $0,1 = \frac{1}{10}$      \\
        centi            & c                             & $0,01 = \frac{1}{100}$    \\
        milli            & m                             & $0,001 = \frac{1}{1000}$  \\ 
    \end{tabular}
    \end{center}
\end{frame}

\begin{frame}
    \frametitle{Prefikser: Noen flere}
    \begin{center}
    \begin{tabular}{l l l}
        \textbf{prefiks} & \textbf{forkortelse / symbol} & \textbf{verdi}            \\ \hline
        Terra            & T                             & 1 000 000 000 000         \\
        Giga             & G                             & 1 000 000 000             \\
        Mega             & M                             & 1 000 000                 \\ \hline
        Kilo             & k                             & 1000                      \\
        Hekto            & h                             & 100                       \\
        deka             & da                            & 10                        \\
                         &                               & 1                         \\
        desi             & d                             & $0,1 = \frac{1}{10}$      \\
        centi            & c                             & $0,01 = \frac{1}{100}$    \\
        milli            & m                             & $0,001 = \frac{1}{1000}$  \\ \hline
        mikro            & $\mu$                         & $0,000 001 = \frac{1}{1 000 000} \\
        nano             & n                             & $0,000 000 001 = \frac{1}{1 000 000 000} \\
    \end{tabular}
    \end{center}
\end{frame}

\begin{frame}
    \frametitle{Prefiks: gjøre om til meter}
Ifølge tabellen har vi at centi er en hundredel. Vi kan bruke dette til å gjøre om til
grunnenheten for lengde, meter:
    \begin{eksempel}
        Skriv 543\text{cm} som meter: \newline
        \textit{Løsning:} Vi bytter ut c med $0,01$:
        \begin{align*}
            543\text{cm} & = 543\cdot0,01\cdot \text{m} \\
        \end{align*}
    \end{eksempel}
\end{frame}

\begin{frame}
    \frametitle{Prefiks: gjøre om til meter}
Ifølge tabellen har vi at centi er en hundredel. Vi kan bruke dette til å gjøre om til
grunnenheten for lengde, meter:
    \begin{eksempel}
        Skriv 543\text{cm} som meter: \newline
        \textit{Løsning:} Vi bytter ut c med $0,01$:
        \begin{align*}
            543\text{cm} & =  543\cdot0,01\cdot \text{m} \\
            & =  5,43\text{m}
        \end{align*}
    \end{eksempel}
    \pause
    \begin{oppgave}[Gjør om til meter:]
        \begin{tabular}{l l l l l}
            a) 247\text{cm} & b) 6420\text{mm} & c) 2,34\text{km} & d) 4,3\text{km} & e) 1,5\text{cm}
        \end{tabular}
    \end{oppgave}
\end{frame}

\begin{frame}
    \begin{center}
        \begin{tikzpicture}[xscale=1, yscale=0.7]
        \draw (0, 0) -- (8,0) -- (8, 1) -- (0,1) -- cycle;
        \draw (1,1) -- (1,0);
        \draw (2,1) -- (2,0);
        \draw (3,1) -- (3,0);
        \draw (4,1) -- (4,0);
        \draw (5,1) -- (5,0);
        \draw (6,1) -- (6,0);
        \draw (7,1) -- (7,0);

        \draw (0.5, 0.5) node {mm};
        \draw (1.5, 0.5) node {cm};
        \draw (2.5, 0.5) node {dm};
        \draw (3.5, 0.5) node {1m};
        \draw[blue] (4.5, 0.5) node {\tiny da m};
        \draw[blue] (5.5, 0.5) node {\tiny hm};
        \draw (6.5, 0.5) node {km};
        \draw (7.5, 0.5) node {mil};

\pause

        \draw[hpil] (0.5, 1) to [out=40, in=140] (1.5, 1);
        \draw (1, 1.4) node {\tiny$\text{:}10$};

\pause
        \draw[hpil] (1.5, 1) to [out=40, in=140] (2.5, 1);
        \draw (2, 1.4) node {\tiny$\text{:}10$};


\pause
        \draw[hpil] (2.5, 1) to [out=40, in=140] (3.5, 1);
        \draw[hpil] (3.5, 1) to [out=40, in=140] (4.5, 1);
        \draw[hpil] (4.5, 1) to [out=40, in=140] (5.5, 1);
        \draw[hpil] (5.5, 1) to [out=40, in=140] (6.5, 1);
        \draw[hpil] (6.5, 1) to [out=40, in=140] (7.5, 1);


        \draw (3, 1.4) node {\tiny$\text{:}10$};
        \draw (4, 1.4) node {\tiny$\text{:}10$};
        \draw (5, 1.4) node {\tiny$\text{:}10$};
        \draw (6, 1.4) node {\tiny$\text{:}10$};
        \draw (7, 1.4) node {\tiny$\text{:}10$};

\pause  
            \draw[hpil] (3.5,1.2) to [out=60, in=110] (6.5, 1.2);
            \draw (5,2.3) node {\tiny$\text{:}1000$};

\pause 
        \draw[hpil] (7.5, 0) to [out=-150, in=-40] (6.5, 0);
        \draw (7, -0.4) node {\tiny$\cdot10$};



\pause
        \draw[hpil] (6.5, 0) to [out=-150, in=-40] (5.5, 0);
        \draw[hpil] (5.5, 0) to [out=-150, in=-40] (4.5, 0);
        \draw[hpil] (4.5, 0) to [out=-150, in=-40] (3.5, 0);
        \draw[hpil] (3.5, 0) to [out=-150, in=-40] (2.5, 0);
        \draw[hpil] (2.5, 0) to [out=-150, in=-40] (1.5, 0);
        \draw[hpil] (1.5, 0) to [out=-150, in=-40] (0.5, 0);

        \draw (6, -0.4) node {\tiny$\cdot10$};
        \draw (5, -0.4) node {\tiny$\cdot10$};
        \draw (4, -0.4) node {\tiny$\cdot10$};
        \draw (3, -0.4) node {\tiny$\cdot10$};
        \draw (2, -0.4) node {\tiny$\cdot10$};
        \draw (1, -0.4) node {\tiny$\cdot10$};

\pause  
            \draw[hpil] (6.5,-0.2) to [out=-110, in=-60] (3.5, -0.2);
            \draw (5,-1.4) node {\tiny$\cdot1000$};


    \end{tikzpicture}
    \end{center}
    \begin{eksempel}
        Skriv 5,45\text{mm} som desimeter (dm). \newline
        Skriv 1,56dm som millimeter (mm) \newline
        Skriv 50 000 \text{cm} som meter (m) \newline
        Skriv 75 000 \text{cm} som kilometer (km).
    \end{eksempel}
\end{frame}


\begin{frame}
    \begin{oppgave}[Gjør om til millimeter:]
        \begin{tabular}{l l l l l}
            a) 0,837\text{cm} & b) 5,4\text{m} & c) 439\text{cm} & d) 4,34\text{dm}
        \end{tabular}
    \end{oppgave}
    \pause
    \begin{oppgave}[Gjør om til centimeter:]
        \begin{tabular}{l l l l l}
            a) 0,125\text{mm} & b) 1,23\text{m} & c) 439\text{dm} & d) 0,6\text{km}
        \end{tabular}
    \end{oppgave}
    \pause
    \begin{oppgave}[Gjør om til kilometer:]
        \begin{tabular}{l l l l l}
            a) 600\text{m} & b) 1230\text{m} & c) 25 000\text{cm} & d) 300 000\text{cm} \\ e) 45 mil
        \end{tabular}
    \end{oppgave}
\end{frame}

\end{document}
